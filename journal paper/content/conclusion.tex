\documentclass[../competing_bandits.tex]{subfiles}
\begin{document}

\section{Conclusion}\label{sec:conclusion}

In this paper we studied the tension between exploration and competition. We considered a stylized duopoly model in which two firms face an identical multi-armed bandit problem and compete for a stream of users. The firms make progress on their learning problem if and only if they are able to attract a user to their platform over their competitors. We investigated two variants of this problem. The first was an analytical variant, \TheoryModel, where the users choose between the firms according to their Bayesian expected reward of visiting the firm's platform. The second we evaluated via numerical simulation, \ExptsModel, where users choose between the firms based on a reputation score determined by the average reward of the recent users who have utilized that firm.


We find that, in both variants, firms are incentivized to adopt a greedy algorithm which does no purposeful exploration and leads to welfare losses for users. We then consider two relaxations of competition. In the first, we relax the rationality of the users and, in the second, we give one of the firms a first-mover advantage. Both of these relaxations of competition induce firms to adopt ``better" bandit algorithms and increase user welfare.


Our model and its implications have two economic interpretations. The first is that, interpreting the adoption of ``better" bandit algorithms as firm innovation, our results can be framed in terms of the classic inverted-U relationship between innovation and competition. However, unlike other models in the literature, the reason that firms do not ``innovate" is not due to the direct costs of innovation. Rather, it is since ``better" algorithms engage in exploration which has negative short-term consequences on user perception of the firms' quality and thus dis-incentivizes the firm to adopt such technologies under stark competition.


The second economic interpretation is that our model speaks to the role that data plays in the digital economy. We find a ``death spiral" effect where exploration by a firm degrades its performance relative to competitors who keep learning and improving from \emph{their} users, and so forth until the platform is completely starved of users. Furthermore, we find that even a small amount of initial data advantage for a firm gets amplified due to the dynamics of competition and leads to that firm acquiring a large market share. Thus, our results highlight that competition endogenously generates the data observed by firms, which naturally leads to results similar to those in markets with network effects and underscores that data can serve as a barrier to entry in the digital economy.

There are several interesting directions to explore for future work. The model we consider in this paper is well-suited as a model for competition between ad-supported platforms who compete primarily on quality and whose services are provided without cost to users. However, it would be interesting to consider the role that giving the firms to ability to set prices would play as this could potentially enable the firms to compensate users in the earlier periods while it is engaging in exploration and prevent such ``death spiral" effects. The second interesting extension would be to consider a setting with horizontally differentiated user preferences\footnote{This would be similar to the extension in Section \ref{sec:theory-extensions} that considers contextual bandits, though it would be interesting to study how this would change based on different user models that had context-specific user decision rules.} as such a model would be better suited for understanding how competition may induce firms to specialize. In general, there are several further directions to undertake in order to better characterize the intertwined relationship between competition and data generation that can help us better understand competition in the digital economy.

\end{document}
%%% Local Variables:
%%% mode: latex
%%% TeX-master: "../competing_bandits"
%%% End:
