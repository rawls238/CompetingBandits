\documentclass[10pt]{article}

\usepackage{color,
setspace,sectsty,comment,footmisc,caption,
pdflscape,subcaption,array,hyperref,adjustbox,threeparttable
}

\usepackage{fullpage}
%\usepackage{geometry}
%\geometry{left=1.0in,right=1.0in,top=1.0in,bottom=1.0in}


\usepackage{subfiles}
\usepackage{libertine}
\usepackage[title,titletoc]{appendix}
\usepackage{url}            % simple URL typesetting
\usepackage{booktabs}       % professional-quality tables
\usepackage{nicefrac}       % compact symbols for 1/2, etc.
\usepackage{microtype}      % microtypography
\usepackage{amsmath,amssymb,amsfonts,amsthm}
\usepackage[round]{natbib} % cannot use with AAAI! A.S.

\usepackage{float}
\usepackage{sgame, tikz} % Game theory packages
\usepackage{algorithm,algpseudocode}
\usepackage{makecell}
 \usepackage{multirow}
 \usepackage{graphicx}

\usepackage{booktabs} % For formal tables


%\theoremstyle{definition}
%\newtheorem{definition}{Definition}
\newtheorem{finding}{Finding}
%\newtheorem{conjecture}{Conjecture}
\newtheorem{puzzle}{Puzzle}
\captionsetup{font=footnotesize}


\usepackage{slivkins-setup,slivkins-theorems}

% model variants
\newcommand{\TheoryModel}{Bayesian-choice model\xspace}
\newcommand{\ExptsModel}{reputation-choice model\xspace}

% Commands from EC 19 paper
\newcommand{\term}[1]{\ensuremath{\mathtt{#1}}\xspace}

% shorthand for algorithms
\newcommand{\TS}{\term{TS}}    % Thompson Sampling
\newcommand{\DEG}{\term{BEG}}  % "Dynamic eps-greedy"
\newcommand{\DG}{\term{BG}}    % "Dynamic Greedy"

% shorthand for models
\newcommand{\HMR}{\term{HMR}} % HardMaxRandom
\newcommand{\HM}{\term{HM}}    % HardMax



\newcommand{\Beta}{\term{Beta}} % for Beta distribution
\newcommand{\Eeog}{\term{EoG}} % shorthand for "effective end of game"

\newcommand{\MRV}{mean reward vector\xspace} % mean reward vector
\newcommand{\MRVs}{mean reward vectors\xspace} % mean reward vector


% commands from "theory"

% in connection with "a little greedy goes a long way"
%     stuff for the "greedy modification".
\newcommand{\gr}{\texttt{gr}}
\newcommand{\BIRgr}{\BIR^{\gr}}
\newcommand{\alggr}{\alg[\gr]}
\newcommand{\rewgr}{\rew_{\gr}}


\DeclareMathOperator*{\Expectation}{\mathbb{E}}
\newcommand{\Ex}[2]{\Expectation_{#1}\left[#2\right]}
\newcommand{\FiniteGame}{finite competition game\xspace}

% notation
%%% advanced notation
\newcommand{\OPT}{\term{OPT}}
\newcommand{\rew}{\term{rew}}  % Bayesian-expected reward after n local rounds
\newcommand{\PMR}{\term{PMR}} % posterior mean reward
\newcommand{\support}{\term{support}}

\newcommand{\BIR}{\term{BIR}} % Bayesian Instantaneous Regret

% response function
\newcommand{\respF}{f_{\term{resp}}}
\newcommand{\respEps}{\eps_\term{resp}}

\newcommand{\BReg}{\term{BReg}}
\newcommand{\HardMax}{\term{HardMax}}
\newcommand{\HardMaxRandom}{\term{HardMax\&Random}}
\newcommand{\SoftMaxRandom}{\term{SoftMax}}
\newcommand{\Uniform}{\term{Uniform}}
\newcommand{\Random}{\term{Random}}

% algorithms
\newcommand{\StaticGreedy}{\term{StaticGreedy}}
\newcommand{\DynGreedy}{\term{BayesianGreedy}}
\newcommand{\DynamicGreedy}{\DynGreedy}
\newcommand{\DynamicEpsGreedy}{\term{BayesianEpsilonGreedy}}
\newcommand{\Thompson}{\term{ThompsonSampling}}

% "monotone algorithms"
\newcommand{\bmonotone}{Bayesian-monotone\xspace}
\newcommand{\bmonotonicity}{Bayesian-monotonicity\xspace}

\newcommand{\termSub}[2]{\ensuremath{\mathtt{#1}_{#2}}\xspace}
\newcommand{\alg}[1][]{\termSub{alg}{#1}}
%\newcommand{\prin}[1][]{\termSub{prin}{#1}}  % principal
\newcommand{\agent}[1][]{\termSub{agent}{#1}}

% priors and posteriors
\newcommand{\prior}{\ensuremath{\mP}\xspace}
\newcommand{\priorMu}{\ensuremath{\prior_\mathtt{mean}}\xspace}
\newcommand{\posteriorN}[2]{\mN_{#1,#2}}  % \posteriorN{principal}{round}

% rationality / innovation / competition
\newcommand{\termTXT}[1]{{\em {#1}}\xspace}

\newcommand{\rationality}{\termTXT{rationality}}
\newcommand{\Rationality}{\termTXT{Rationality}}
\newcommand{\innovation}{\termTXT{innovation}}
\newcommand{\Innovation}{\termTXT{Innovation}}
\newcommand{\competition}{\termTXT{competition}}
\newcommand{\Competition}{\termTXT{Competition}}
\newcommand{\competitiveness}{\termTXT{competitiveness}}
\newcommand{\exploration}{\termTXT{exploration}}
\newcommand{\Exploration}{\termTXT{Exploration}}

% a very useful package for edits and comments, from David Kempe (USC)
\usepackage[suppress]{color-edits}
%\usepackage[suppress]{color-edits}  % use this to suppress the package
\addauthor{as}{red}    % as for Alex
\addauthor{ga}{blue}  % ga for Guy
\addauthor{sw}{orange} % sw for Steven
\addauthor{ym}{madenta} % sw for Steven
% e.g. for Alex, provides \asedit{}, \ascomment{} and \asdelete{}.

\setlength{\tabcolsep}{8pt} % Default value: 6pt
\renewcommand{\arraystretch}{1.5} % Default value: 1



\begin{document}

%\begin{titlepage}

\title{Competing Bandits:\\
The Perils of Exploration under Competition%
\thanks{This is an extended abstract of \citet{CompetingBandits-merged}, which in turn subsumes conference papers
\citet{CompetingBandits-itcs18} and \citet{CompetingBandits-ec19}.}}
% \textit{Competing Bandits: Learning under Competition}, at Innovations in Theoretical Computer Science 2018, and \textit{The Perils of Exploration under Competition: A Computational Modeling Approach}, at ACM Economics and Computation 2019.}}

\author{Guy Aridor%
\footnote{Columbia University, Department of Economics. Email: g.aridor@columbia.edu}
\and
%\rule{0.2in}{0pt}
Yishay Mansour%
\footnote{Google and Tel Aviv University, Department of Computer Science. Email: mansour.yishay@gmail.com}
\and
%\rule{0.2in}{0pt}
Aleksandrs Slivkins%
\footnote{Microsoft Research New York City. Email: slivkins@microsoft.com}
\and
%\rule{0.2in}{0pt}
Zhiwei Steven Wu%
\footnote{Carnegie-Mellon University, Pittsburgh, PA.
Email: zstevenwu@cmu.edu. 
Partially done as an intern and a postdoc at MSR-NYC.
}}
\date{August 2020}
\maketitle

Learning from interactions with users is ubiquitous in modern customer-facing platforms, from product recommendations to web search to content selection to fine-tuning user interfaces. Many platforms purposefully implement \emph{exploration}: making potentially suboptimal choices for the sake of acquiring new information.
Online platforms routinely deploy A/B tests, and are increasingly adopting  more sophisticated exploration methodologies based on \emph{multi-armed bandits}, a standard and well-studied framework for exploration and making decisions under uncertainty
\citep{Gittins-book11,Bubeck-survey12,slivkins-MABbook,LS19bandit-book}.

%~\cite{KohaviAB-2015,KohaviLSH09}

In this paper, we initiate a study of the interplay between \exploration and \competition. Platforms that engage in exploration typically need to compete against one another; most importantly, they compete for users. This creates a tension:
%between \exploration and \competition.
while exploration may be essential for improving the service tomorrow, it may degrade quality and make users leave \emph{today}, in which case there will be less users to learn from. This may further degrade the platform's performance relative to competitors who keep learning and improving from \emph{their} users, and so forth. Taken to the extreme, such dynamics may create a ``death spiral" effect when the vast majority of customers eventually switch to competitors. Users therefore serve three distinct roles: they are customers that generate revenue, they are sources of data for learning, and they are self-interested agents who choose among the competing systems.

The main high-level question is:
%\begin{align}\label{eq:main-Q}
\textbf{How does competition incentivize the adoption of better exploration algorithms?}
%\end{align}
This translates into a number of more concrete questions. While it is commonly assumed that better technology always helps, is this so under competition? Does increased competition lead to higher consumer welfare? How significant are ``data feedback loops" --- when more data leads to more users, which leads to even more data, etc. --- and how they relate to the anti-trust considerations?
%Finding formalizations that admit meaningful answers is a major part of the overall challenge.
To answer these questions, we study complex interactions between platforms' learning dynamics and users' self-interested behavior. The choice of a particular technology (exploration algorithm) is no longer an abstract, static choice with a predetermined outcome for the platform. Instead, we model the algorithms explicitly, and investigate how they play out in competition over an extended period of time.
We offer a mix of theoretical results and numerical simulations, in a range of models.


%\footnote{This is a fundamental question which is part of a larger policy discussion around whether data can serve as an indirect network effect and lead to similar ``market tipping" results as is standard in the literature on competition in markets with network effects (see \cite{jullien2019economics} for a policy oriented discussion of this).}

% the extent to which the game between the two principals is competitive
% degree of innovation that these models incentivize.
% the extent to which agents make rational decisions


%\subsection{Our model}
%\label{sec:intro-model}

%\subsection*{Our Model: competing bandits}
%\label{sec:intro-model}

%We investigate these questions with
\xhdr{Model: competition game.} We consider a stylized duopoly model where two firms commit to exploration strategies and compete for a stream of consumers. We define a game in which two firms (\emph{principals}) simultaneously engage in exploration and compete for users (\emph{agents}). These two processes are interlinked, as exploration decisions are experienced by users and informed by their feedback. We need to specify several conceptual pieces: how the principals and agents interact, what is the machine learning problem faced by each principal, and what is the information structure. Each piece can get rather complicated in isolation, let alone jointly, so we strive for simplicity.
%Thus, the key features of our model are as follows.
%\begin{itemize}

\textbf{(i)} 
A new agent arrives in each round $t=1,2, \ldots$, and chooses among the two principals. The principal chooses an action (\eg a list of web search results to show to the agent), the user experiences this action, and reports a reward. All agents have the same ``decision rule" for choosing among the principals given the available information.

\textbf{(ii)}
Each principal faces a basic and well-studied version of the multi-armed bandit problem: for each arriving agent, it chooses from a fixed set of actions  (a.k.a. \emph{arms}) and receives a reward drawn independently from a fixed distribution specific to this action. The reward distributions are initially unknown.
% (but can be estimated over time from the data).

\textbf{(iii)}
Principals simultaneously announce their bandit algorithms before round $1$, and cannot change them afterwards.  Each principal's objective is to maximize its market share (the fraction of users choosing this principal). Each principal only observes agents that chose this principal.
%, but may have access to each platforms' reputation score (more on this in    Section~\ref{sec:intro-discussion}).
%\end{itemize}

We investigate several model variants, as we vary agents' decision rule and/or posit a first-mover advantage.

%We consider two model variants, which determine agents' decision rule and their information sets. In \emTheoryModel, there is a common Bayesian prior on the reward distributions. Agents do not receive any other information and choose between the principals using their knowledge of $t$ and the principals' algorithms. In \emExptsModel, agents have access to a reputation score for each principal, which is a sliding window average of the rewards experienced by previous agents that have visited this principal. The former variant used for the theoretical results, the latter for simulations.


\xhdr{Technology: multi-armed bandit algorithms.}
To compare between bandit algorithms, we build on prevalent intuition in the literature. We focus on standard notions of regret, and distinguish between three classes of bandit algorithms: ones that never explicitly explore (\emph{greedy algorithms}), ones that explore without looking at the data (\emph{exploration-separating algorithms}), and ones where exploration gradually zooms in on the best arm (\emph{adaptive-exploration algorithms}). In isolation, \ie in the absence of competition, these classes are fairly well-understood. Greedy algorithms are terrible for a wide variety of problem instances, precisely because they never explore. Exploration-separated algorithms learn at a reasonable but mediocre rate across all problem instances. Adaptive-exploration algorithms are optimal in the worst case, and exponentially improve for ``easy" problem instances. Generally,  ``better" algorithms are better in the long run, but could be worse initially.



%\subsection*{Our Results}
%\label{sec:intro-results}


\xhdr{Theoretical results.}
%We endow agents with Bayesian rationality, a common modeling approach for a theoretical investigation.
We consider a basic Bayesian model, where agents have a common Bayesian prior on reward distributions, and know the principals' algorithms. For tractability, we posit that agents do not receive any information about the previous agents' choices and rewards. Each agent knows the round (s)he arrives in, computes the Bayesian-expected reward for each principal, and use these two numbers to decide which principal to choose.
%We refer to this variant as the \emph{\TheoryModel}. 

Our results depend crucially on the agents' decision rule:
%\begin{itemize}

\textbf{(i)}  
The most obvious decision rule (\HardMax) maximizes the current agent's Bayesian-expected reward. We find that \HardMax is not conducive to adopting better algorithms: each principal's dominant strategy is to choose the greedy algorithm. Further, we show that \HardMax is sensitive to tie-breaking: if the tie-breaking is probabilistically biased in favor of one principal, then this principal has a simple ``winning strategy" no matter what.

\textbf{(ii)} 
We dilute the \HardMax agents with a small fraction of ``random agents" who choose a principal uniformly at random. We call this model \HardMaxRandom. We find that better algorithms help in a big way: a sufficiently better algorithm is guaranteed to win all non-random agents after an initial learning phase. However, there is a substantial caveat: one can defeat any algorithm by interleaving it with the greedy algorithm. This has two undesirable corollaries: a better algorithm may sometimes lose in competition, and a pure Nash equilibrium typically does not exist.

\textbf{(iii)} 
We further relax the decision rule so that the probability of choosing a given principal varies smoothly as a function of the difference between  principals' Bayesian-expected rewards; we call it \SoftMaxRandom. Then, the ``better algorithm wins" result holds under much weaker assumptions on what constitutes a better algorithm. This is the most technical result of the paper. The competition in this setting is necessarily much more relaxed: typically, both principals attract approximately half of the agents as time goes by (but a better algorithm would attract slightly more).

\xhdr{Economic interpretation: the inverted-U relationship.}
Interpreting the adoption of better algorithms as ``innovation", our findings can be framed in terms of the \emph{inverted-U relationship} between competition and innovation.% 
%\footnote{Here, ``innovation" refers to adoption of a better technology at a substantial R\&D expense to a given firm. It is not salient whether similar ideas and/or technologies already exist elsewhere. Adoption of exploration algorithms tends to require substantial R\&D effort in practice, even if the algorithms themselves are well-known in the research literature \citep[\eg see][]{DS-arxiv}.}
\footnote{This is a well-established concept in the economics literature, dating back to \cite{Schumpeter-42}, whereby too little or too much competition is bad for innovation, but intermediate levels of competition tend to be better \citep[\eg][]{aghion2005competition,Vives-08}.}

Our decision rules differ in terms of rationality: from fully rational decisions with \HardMax to relaxed rationality with \HardMaxRandom to an even more relaxed rationality with \SoftMaxRandom. The same distinctions also control the severity of competition between the principals: from cut-throat competition with \HardMax to a more relaxed competition with \HardMaxRandom, to an even more relaxed competition with \SoftMaxRandom. Indeed, with \HardMax you lose all customers as soon as you fall behind in performance, with \HardMaxRandom you get some small market share no matter what, and with \SoftMaxRandom you are further guaranteed a market share close to $\tfrac12$ as long as your performance is not much worse than the competition. The uniform choice among principals corresponds to no rationality and no competition. While agents' rationality and severity of competition are often modeled separately in the literature, it is not unusual to have them modeled with the same ``knob" \cite[\eg][]{Gabaix-16}.


We identify the inverted-U relationship driven by the rationality/competitiveness distinctions outlined above: from \HardMax to \HardMaxRandom to \SoftMaxRandom to \Uniform. We also find another, technically different, inverted-U relationship which zeroes in on the \HardMaxRandom model. We vary rationality/competitiveness inside this model, and track the marginal utility of switching to a better algorithm.

These inverted-U relationships are driven by different aspects in our model than the ones in the existing literature in economics. The latter focuses on the tradeoff between the R\&D costs and the benefits that the improved technology provides in the competition. In our case, the barriers for innovations arise entirely from the reputational consequences of exploration in competition, even in the absence of R\&D costs.



\xhdr{Numerical simulations.}
We consider a basic frequentist model. We posit that the agents observe signals about the principals' past performance, and base their decisions on these signals alone, without invoking any prior knowledge or beliefs. The performance signals are abstracted and aggregated as a scalar \emph{reputation score} for each principal, modeled as a sliding window average of its rewards. Thus, agents' decision rule depends only on the two reputation scores. 
%We refer to this variant as the \emph{\ExptsModel}.%
%\footnote{In comparison, the theoretical results focus on another extreme, with Bayesian rationality and no performance signals.}
We refine and expand the theoretical results in several ways.

%\begin{itemize}
\textbf{(i)}  
We compare \HardMax and \HardMaxRandom decision rules. We find that the greedy algorithm often wins under \HardMax, with a strong evidence of the ``death spiral" effect mentioned earlier.
 As predicted by the theory, better algorithms prevail if the expected number of ``random" users is sufficiently large. However, this effect is negligible for smaller parameter values.

%\footnote{\asedit{Reputation scores already introduce some noise into users' choices. However, the amount of noise due to this channel is typically small, both in our simulations and in practice, because reputation signals average over many datapoints.}}

\textbf{(ii)}  
 We investigate the first-mover advantage as a different channel to vary the intensity of competition: from the first-mover to simultaneous arrival to late-arriver. (We focus on the \HardMax decision rule.) We find that the first-mover is incentivized to choose a more advanced exploration algorithm, whereas the late-arriver is often incentivized to choose the ``greedy algorithm" (more so than under simultaneous arrival). Consumer welfare is higher under early/late arrival than under simultaneous entry. We frame these results in terms of an inverted-U relationship.


%\footnote{\asedit{We consider the ``permanent monopoly" scenario for comparison only, without presenting any findings. We just assume that a monopolist chooses the greedy algorithm, because it is easier to deploy in practice. Implicitly, users have no ``outside option": the service provided is an improvement over not having it (and therefore the monopolist is not incentivized to deploy better learning algorithms). This is plausible with free ad-supported platforms such as Yelp or Google.}}

\textbf{(iii)}  
We decompose the first-mover advantage into two distinct effects: free data to learn from (\emph{data advantage}), and a more definite, and possibly better reputation compared to an entrant (\emph{reputation advantage}). We run additional experiments so as to isolate and compare these two effects. We find that either effect alone leads to a significant advantage under competition. The data advantage is larger than reputation advantage when the incumbent commits to a more advanced bandit algorithm. Finally, we find an ``amplification effect" of the data advantage: even a small amount thereof gets amplified under competition, causing a large difference in eventual market shares.

%\end{itemize}

\xhdr{Economic interpretation: network effects of data.}
Our model speaks to policy discussions on regulating data-intensive digital platforms \citep{furman2019unlocking, scott2019committee}, and particularly to the ongoing debate on the role of data in the digital economy. One fundamental question in this debate is whether data can serve a similar role as traditional ``network effects", creating scenarios when only one firm can function in the market \citep{Rysman09, jullien2019economics}.
%whereby, when these effects are present, in many cases only one firm can function in the market, leading to competition \emph{for} the market being more important than competition \emph{in} the market \citep{Rysman09, jullien2019economics}.
The death spiral/amplification effects mentioned above have a similar flavor as network effects: a relatively small amount of exploration (resp., data advantage)  gets amplified under competition and causes the firm to be starved of users (resp., take over most of the market).
%\gaedit{We further find that a small data advantage for one firm gets amplified under competition and leads to that firm taking the entire market, showing that data can provide a similar incumbency advantage as those provided by traditional network effects and can serve as a barrier to entry in online markets.}
A distinctive feature of our approach is that we explicitly model the learning problem of the firms and consider them deploying algorithms for solving this problem.  Thus, we do not explicitly model the network effects, but they arise endogenously from our setup.

Our results highlight that understanding the performance of learning algorithms in isolation does not necessarily translate to understanding their impact in competition, precisely due to the fact that competition leads to the endogenous generation of data observed by the firms. Approaches such as \citet{lambrecht2015can, bajari2018impact, varian2018artificial} argue that the diminishing returns to scale and scope of data in isolation mitigate such data feedback loops,
%as non-existent
but ignore the differences induced by learning in isolation versus under competition. Furthermore, explicitly incorporating the interaction between learning technology and data creation allows us to speak on how data advantages are characterized and amplified not only by data quantity, but also the increased data quality gathered by better learning algorithms.

\xhdr{Significance.}
Our theory takes a Bayesian approach and discovers several strong asymptotic results. The numerical simulations provide a more nuanced and ``non-asymptotic" perspective. In essence, we look for substantial effects within relevant time scales. In fact, we start our investigation by determining what time scales are relevant in the context of our model. Our study has a dual purpose: (i) shed light on real-world implications of some typical scenarios, and (ii) investigate the space of models for describing the real world. As an example to clarify the latter point, consider the \HardMax model with simultaneous entry. It is is not necessarily the most realistic model. However, our results elucidate the need for more refined models that allow for ``free exploration" (\eg via random agents or early entry).



\newpage

%
% The code below should be generated by the tool at
% http://dl.acm.org/ccs.cfm
% Please copy and paste the code instead of the example below.
%

\addcontentsline{toc}{section}{References}

\bibliographystyle{plainnat}
%\begin{small}
\bibliography{bib-abbrv,bib-ML,refs,bib-bandits,bib-AGT,bib-slivkins, bib-random}
%\end{small}

\clearpage
%\subfile{content/old-intro}

\end{document}
%%% Local Variables:
%%% mode: latex
%%% TeX-master: t
%%% End:
