Most online platforms learn from interactions with users, and engage in \emph{exploration}: making potentially suboptimal choices in order to acquire new information. We study the interplay between \emph{exploration} and \emph{competition}: how such platforms balance the exploration for learning and competition for users.
%Here users play three distinct roles: they are customers that generate revenue, they are sources of data for learning, and they are self-interested agents which choose among the competing platforms.

We consider a stylized duopoly in which two firms face the same multi-armed bandit problem. Users arrive one by one and choose between the two firms, so that each firm makes progress on its bandit problem only if it is chosen. We study whether competition incentivizes the adoption of better algorithms. We find that stark competition disincentivizes exploration, leading to low welfare.
%induces firms to commit to a ``greedy" algorithm that leads to low welfare.
However, weaker competition
%by providing firms with some ``free" users
incentivizes better exploration algorithms and increases welfare.  We investigate two channels for weakening the competition: stochastic user choice models and a first-mover advantage.
Our findings speak to the competition-innovation relationship and the first-mover advantage in the digital economy.
%Our findings are closely related to the ``competition vs. innovation" relationship, and elucidate the first-mover advantage in the digital economy.
