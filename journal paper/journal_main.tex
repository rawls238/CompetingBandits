\documentclass[format=acmsmall, review=false]{acmart}

\usepackage{ec19paper/acm-ec-19}
\usepackage{itcs18paper/slivkins-setup,itcs18paper/slivkins-theorems}
\usepackage{tikz} % SW: Added for plots
\usepackage{subfiles}
\usepackage[toc,page]{appendix}
\usepackage{url}            % simple URL typesetting
\usepackage{booktabs}       % professional-quality tables
\usepackage{nicefrac}       % compact symbols for 1/2, etc.
\usepackage{microtype}      % microtypography
\usepackage{amsmath,amssymb,amsfonts}
%\usepackage{natbib} % cannot use with AAAI! A.S.

\usepackage{amsthm} % The amsthm package provides extended theorem environments
\usepackage{float}
\usepackage{sgame, tikz} % Game theory packages
\usepackage{caption}
\usepackage{algorithm,algpseudocode}
\usepackage{makecell}
 \usepackage{multirow}
 \usepackage{graphicx}
\theoremstyle{definition}
%\newtheorem{definition}{Definition}
\newtheorem{finding}{Finding}
%\newtheorem{conjecture}{Conjecture}
\newtheorem{puzzle}{Puzzle}
\captionsetup{font=footnotesize}

% Commands from EC 19 paper
\newcommand{\term}[1]{\ensuremath{\mathtt{#1}}\xspace}
\newcommand{\TS}{\term{TS}}
\newcommand{\DEG}{\term{DEG}}
\newcommand{\DG}{\term{DG}}
\newcommand{\AD}{\term{AD}}
\newcommand{\EGR}{\term{EGR}}
\newcommand{\GR}{\term{GR}}
\newcommand{\Beta}{\term{Beta}} % for Beta distribution
\newcommand{\Eeog}{\term{EoG}} % shorthand for "effective end of game"

\newcommand{\MRV}{mean reward vector\xspace} % mean reward vector
\newcommand{\MRVs}{mean reward vectors\xspace} % mean reward vector

\newcommand{\HMR}{\term{HMR}}
\newcommand{\HM}{\term{HM}}


% commands from ITCS 18 paper

\DeclareMathOperator*{\Expectation}{\mathbb{E}}
\newcommand{\Ex}[2]{\Expectation_{#1}\left[#2\right]}


% notation
%%% advanced notation
\newcommand{\OPT}{\term{OPT}}
\newcommand{\rew}{\term{rew}}  % Bayesian-expected reward after n local rounds
\newcommand{\REP}{\term{REP}} % posterior mean reward
\newcommand{\PMR}{\term{PMR}} % posterior mean reward
\newcommand{\support}{\term{support}}

\newcommand{\BIR}{\term{BIR}} % Bayesian Instantaneous Regret
\newcommand{\regret}{R^{\term{inst}}} % regret
\newcommand{\regretWC}{\regret_{\term{wc}}} % regret

% response function
\newcommand{\respF}{f_{\term{resp}}}
\newcommand{\respEps}{\eps_\term{resp}}

\newcommand{\BReg}{\term{BReg}}
\newcommand{\HardMax}{\term{HardMax}}
\newcommand{\HardMaxRandom}{\term{HardMax\&Random}}
\newcommand{\SoftMaxRandom}{\term{SoftMax}}
\newcommand{\Uniform}{\term{Uniform}}
\newcommand{\Random}{\term{Random}}

\newcommand{\StaticGreedy}{\term{StaticGreedy}}
\newcommand{\DynGreedy}{\term{DynamicGreedy}}
\newcommand{\DynamicGreedy}{\term{DynamicGreedy}}

\newcommand{\termSub}[2]{\ensuremath{\mathtt{#1}_{#2}}\xspace}
\newcommand{\alg}[1][]{\termSub{alg}{#1}}
%\newcommand{\prin}[1][]{\termSub{prin}{#1}}  % principal
\newcommand{\agent}[1][]{\termSub{agent}{#1}}

% priors and posteriors
\newcommand{\prior}{\ensuremath{\mP}\xspace}
\newcommand{\priorMu}{\ensuremath{\prior_\mathtt{mean}}\xspace}
\newcommand{\posteriorN}[2]{\mN_{#1,#2}}  % \posteriorN{principal}{round}

% rationality / innovation / competition
\newcommand{\termTXT}[1]{{\em {#1}}\xspace}

\newcommand{\rationality}{\termTXT{rationality}}
\newcommand{\Rationality}{\termTXT{Rationality}}
\newcommand{\innovation}{\termTXT{innovation}}
\newcommand{\Innovation}{\termTXT{Innovation}}
\newcommand{\competition}{\termTXT{competition}}
\newcommand{\Competition}{\termTXT{Competition}}
\newcommand{\competitiveness}{\termTXT{competitiveness}}
\newcommand{\exploration}{\termTXT{exploration}}
\newcommand{\Exploration}{\termTXT{Exploration}}
% a very useful package for edits and comments, from David Kempe (USC)
\usepackage[suppress]{ec19paper/color-edits}
%\usepackage[suppress]{color-edits}  % use this to suppress the package
\addauthor{as}{red}    % as for Alex
\addauthor{ga}{blue}  % ga for Guy
\addauthor{sw}{orange} % sw for Steven
% e.g. for Alex, provides \asedit{}, \ascomment{} and \asdelete{}.
\setlength{\tabcolsep}{8pt} % Default value: 6pt
\renewcommand{\arraystretch}{1.5} % Default value: 1

%drt24 hacks
% Letter paper

% Hack to try to make acmart work with biblatex: https://tex.stackexchange.com/questions/37076/is-it-possible-to-load-biblatex-with-a-class-that-has-already-loaded-natbib
\let\citename\relax
%\RequirePackage[abbreviate=true, dateabbrev=true, isbn=true, doi=true, urldate=comp, url=true, maxbibnames=9, backref=false, backend=biber, style=ACM-Reference-Format, language=american]{biblatex}



\usepackage{booktabs} % For formal tables


% Copyright
%\setcopyright{none}
%\setcopyright{acmcopyright}
%\setcopyright{acmlicensed}
\setcopyright{rightsretained}
%\setcopyright{usgov}
%\setcopyright{usgovmixed}
%\setcopyright{cagov}
%\setcopyright{cagovmixed}


\begin{document}
\title[Competing Bandits: \\ The Perils of Exploration under Competition]
{Competing Bandits: \\ The Perils of Exploration under Competition}


 \author{Guy Aridor}
 \affiliation{ \institution{Columbia University}}
 \author{Aleksandrs Slivkins}
 \affiliation{\institution{Microsoft Research, New York City}}
 \author{Zhiwei Steven Wu}
 \affiliation{\institution{University of Minnesota - Twin Cities}}
 
 \authorsaddresses{%
  Authors' addresses: Guy Aridor, Columbia University, New York, NY, g.aridor@columbia.edu; Aleksandrs Slivkins, Microsoft Research, New York, NY, slivkins@microsoft.com; Zhiwei Steven Wu, University of Minnesota - Twin Cities, Minneapolis, MN, zsw@umn.edu}

\begin{abstract}
\gaedit{Most modern systems strive to learn from interactions with users, and many engage in \emph{exploration}: making potentially suboptimal choices for the sake of acquiring new information. We initiate a study of the interplay between \emph{exploration and competition}---how such systems balance the exploration for learning and the competition for users. Here the users play three distinct roles: they are customers that generate revenue, they are sources of data for learning, and they are self-interested agents which choose among the competing systems.

In our model, we consider competition between two multi-armed bandit algorithms faced with the same bandit instance. Users arrive one by one and choose among the two algorithms, so that each algorithm makes progress if and only if it is chosen. We ask whether and to what extent competition incentivizes the adoption of better bandit algorithms. We investigate this issue analytically for several models of user response, as we vary the degree of competitiveness in the model. We further investigate this issue using numerical simulation to understand the effect of the first-mover advantage in our model. Our findings shed light on the first-mover advantage in the digital economy by exploring the role that data can play as a barrier to entry in online markets and are closely related to the ``competition vs. innovation" relationship, a well-studied theme in economics.}
\end{abstract}

%
% The code below should be generated by the tool at
% http://dl.acm.org/ccs.cfm
% Please copy and paste the code instead of the example below.
%
\maketitle
\keywords{Multi-armed bandits,Exploration,Competition,Competition vs Innovation}


\section{Introduction}
\label{sec:intro}
\subfile{content/sec-intro}

\section{Related work}
\label{sec:related-work}
\subfile{content/related_work}

\section{Our model and preliminaries}
\label{sec:model}
\subfile{content/sec-model}

\section{Full rationality (HardMax)}
\label{sec:rational}
\subfile{itcs18paper/sec-rational}

\section{Relaxed rationality: HardMax \& Random}
\label{sec:random}
\subfile{itcs18paper/sec-random}

\section{SoftMax response function}
\label{sec:soft}
\subfile{itcs18paper/sec-soft}

%\section{Economic implications}
%\label{sec:welfare}
%\subfile{itcs18paper/sec-welfare}

\gaedit{
\subfile{content/sim_details}
}

\subfile{ec19paper/content/perf_in_iso}

\subfile{ec19paper/content/inverted_u}

\subfile{ec19paper/content/barriers}

\subfile{ec19paper/content/revisited}

\gaedit{
\subfile{content/conclusion}
}
%\subfile{ec19paper/content/non_greedy_choice}

\newpage
\bibliographystyle{acm}
\bibliography{bib-ML,refs,bib-abbrv-short,bib-bandits,bib-AGT,bib-slivkins}

\begin{appendices}

\section{Background on multi-armed bandits}
\label{app:examples}
\subfile{itcs18paper/app-examples}

\section{Non-degeneracy via a random perturbation}
\label{app:perturb}
\subfile{itcs18paper/app-perturb}
\subfile{ec19paper/content/appendix_for_one_version}

\end{appendices}
\end{document}
%%% Local Variables:
%%% mode: latex
%%% TeX-master: t
%%% End:
