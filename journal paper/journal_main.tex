\documentclass[12pt]{article}

\usepackage{amssymb,amsmath,amsfonts,
float,geometry,ulem,graphicx,caption,color,
setspace,sectsty,comment,footmisc,caption,natbib,
pdflscape,subcaption,array,hyperref, subfiles,adjustbox,threeparttable
}
\usepackage{tikz} % SW: Added for plots
\usepackage{subfiles}
\usepackage{libertine}
\usepackage[toc,page]{appendix}
\usepackage{url}            % simple URL typesetting
\usepackage{booktabs}       % professional-quality tables
\usepackage{nicefrac}       % compact symbols for 1/2, etc.
\usepackage{microtype}      % microtypography
\usepackage{amsmath,amssymb,amsfonts}
\usepackage{natbib} % cannot use with AAAI! A.S.

\usepackage{amsthm} % The amsthm package provides extended theorem environments
\usepackage{float}
\usepackage{sgame, tikz} % Game theory packages
\usepackage{algorithm,algpseudocode}
\usepackage{makecell}
 \usepackage{multirow}
 \usepackage{graphicx}
\theoremstyle{definition}
%\newtheorem{definition}{Definition}
\newtheorem{finding}{Finding}
%\newtheorem{conjecture}{Conjecture}
\newtheorem{puzzle}{Puzzle}
\captionsetup{font=footnotesize}


\usepackage{slivkins-setup,slivkins-theorems}

% Commands from EC 19 paper
\newcommand{\term}[1]{\ensuremath{\mathtt{#1}}\xspace}
\newcommand{\TS}{\term{TS}}
\newcommand{\DEG}{\term{DEG}}
\newcommand{\DG}{\term{DG}}
\newcommand{\AD}{\term{AD}}
\newcommand{\EGR}{\term{EGR}}
\newcommand{\GR}{\term{GR}}
\newcommand{\Beta}{\term{Beta}} % for Beta distribution
\newcommand{\Eeog}{\term{EoG}} % shorthand for "effective end of game"

\newcommand{\MRV}{mean reward vector\xspace} % mean reward vector
\newcommand{\MRVs}{mean reward vectors\xspace} % mean reward vector

\newcommand{\HMR}{\term{HMR}}
\newcommand{\HM}{\term{HM}}


% commands from ITCS 18 paper

\DeclareMathOperator*{\Expectation}{\mathbb{E}}
\newcommand{\Ex}[2]{\Expectation_{#1}\left[#2\right]}


% notation
%%% advanced notation
\newcommand{\OPT}{\term{OPT}}
\newcommand{\rew}{\term{rew}}  % Bayesian-expected reward after n local rounds
\newcommand{\REP}{\term{REP}} % posterior mean reward
\newcommand{\PMR}{\term{PMR}} % posterior mean reward
\newcommand{\support}{\term{support}}

\newcommand{\BIR}{\term{BIR}} % Bayesian Instantaneous Regret
\newcommand{\regret}{R^{\term{inst}}} % regret
\newcommand{\regretWC}{\regret_{\term{wc}}} % regret

% response function
\newcommand{\respF}{f_{\term{resp}}}
\newcommand{\respEps}{\eps_\term{resp}}

\newcommand{\BReg}{\term{BReg}}
\newcommand{\HardMax}{\term{HardMax}}
\newcommand{\HardMaxRandom}{\term{HardMax\&Random}}
\newcommand{\SoftMaxRandom}{\term{SoftMax}}
\newcommand{\Uniform}{\term{Uniform}}
\newcommand{\Random}{\term{Random}}

\newcommand{\StaticGreedy}{\term{StaticGreedy}}
\newcommand{\DynGreedy}{\term{DynamicGreedy}}
\newcommand{\DynamicGreedy}{\term{DynamicGreedy}}

\newcommand{\termSub}[2]{\ensuremath{\mathtt{#1}_{#2}}\xspace}
\newcommand{\alg}[1][]{\termSub{alg}{#1}}
%\newcommand{\prin}[1][]{\termSub{prin}{#1}}  % principal
\newcommand{\agent}[1][]{\termSub{agent}{#1}}

% priors and posteriors
\newcommand{\prior}{\ensuremath{\mP}\xspace}
\newcommand{\priorMu}{\ensuremath{\prior_\mathtt{mean}}\xspace}
\newcommand{\posteriorN}[2]{\mN_{#1,#2}}  % \posteriorN{principal}{round}

% rationality / innovation / competition
\newcommand{\termTXT}[1]{{\em {#1}}\xspace}

\newcommand{\rationality}{\termTXT{rationality}}
\newcommand{\Rationality}{\termTXT{Rationality}}
\newcommand{\innovation}{\termTXT{innovation}}
\newcommand{\Innovation}{\termTXT{Innovation}}
\newcommand{\competition}{\termTXT{competition}}
\newcommand{\Competition}{\termTXT{Competition}}
\newcommand{\competitiveness}{\termTXT{competitiveness}}
\newcommand{\exploration}{\termTXT{exploration}}
\newcommand{\Exploration}{\termTXT{Exploration}}

% a very useful package for edits and comments, from David Kempe (USC)
\usepackage{color-edits}
%\usepackage[suppress]{color-edits}  % use this to suppress the package
\addauthor{as}{red}    % as for Alex
\addauthor{ga}{blue}  % ga for Guy
\addauthor{sw}{orange} % sw for Steven
% e.g. for Alex, provides \asedit{}, \ascomment{} and \asdelete{}.
\setlength{\tabcolsep}{8pt} % Default value: 6pt
\renewcommand{\arraystretch}{1.5} % Default value: 1

%drt24 hacks
% Letter paper

% Hack to try to make acmart work with biblatex: https://tex.stackexchange.com/questions/37076/is-it-possible-to-load-biblatex-with-a-class-that-has-already-loaded-natbib
\let\citename\relax
%\RequirePackage[abbreviate=true, dateabbrev=true, isbn=true, doi=true, urldate=comp, url=true, maxbibnames=9, backref=false, backend=biber, style=ACM-Reference-Format, language=american]{biblatex}

\normalem
\raggedbottom
% \usepackage[hidelinks]{hyperref}
\onehalfspacing

\geometry{left=1.0in,right=1.0in,top=1.0in,bottom=1.0in}

\usepackage{booktabs} % For formal tables

\begin{document}

\begin{titlepage}
\title{Competing Bandits: \\ The Perils of Exploration under Competition\thanks{This paper is a merged version of \textit{Competing Bandits: Learning under Competition}, at Innovations in Theoretical Computer Science 2018, and \textit{The Perils of Exploration under Competition: A Computational Modeling Approach}, at ACM Economics and Computation 2019.}}
\author{Guy Aridor\footnote{Columbia University, Department of Economics. Email: g.aridor@columbia.edu} \rule{0.2in}{0pt} Yishay Mansour\footnote{Google and Tel Aviv University, Department of Computer Science. Email: mansour.yishay@gmail.com;} \rule{0.2in}{0pt}   Aleksandrs Slivkins\footnote{Microsoft Research - New York City. Email: slivkins@microsoft.com} \rule{0.2in}{0pt}   Zhiwei Steven Wu\footnote{University of Minnesota - Twin Cities, Department of Computer Science. Email: zsw@umn.edu.}}
\date{\today}
\maketitle
\begin{abstract}
Most modern firms strive to learn from interactions with consumers, and many engage in \emph{exploration}: making potentially suboptimal choices for the sake of acquiring new information. We initiate a study of the interplay between \emph{exploration} and \emph{competition}—how such firms balance the exploration for learning and the competition for consumers. Here consumers play three distinct roles: they are customers that generate revenue, they are sources of data for learning, and they are self-interested agents which choose among the competing firms.

We consider competition between two firms facing the same multi-armed bandit instance who simultaneously choose a bandit algorithm. Users arrive one by one and choose between the two firms, so that each firm makes progress on its bandit instance if and only if it is chosen. We study to what extent competition leads to welfare increases for consumers and what algorithms firms adopt under competition. We find that stark competition induces firms to commit to a ``greedy" algorithm that leads to low consumer welfare. However, we find that weakening competition by providing firms with some ``free" consumers incentivizes better exploration strategies and increases consumer welfare. Our findings shed light on the first-mover advantage in the digital economy by exploring the role that data can play as a barrier to entry in online markets and are closely related to the ``competition vs. innovation" relationship, a well-studied theme in economics.

\vspace{0.2in}

\bigskip
\end{abstract}
\setcounter{page}{0}
\thispagestyle{empty}
\end{titlepage}
\pagebreak \newpage



\tableofcontents

%
% The code below should be generated by the tool at
% http://dl.acm.org/ccs.cfm
% Please copy and paste the code instead of the example below.
%
\section{Introduction}
\label{sec:intro}
\subfile{content/sec-intro}

\section{Related work}
\label{sec:related-work}
\subfile{content/related_work}

\section{Our model and preliminaries}
\label{sec:model}
\subfile{content/sec-model}

\section{Full rationality (HardMax)}
\label{sec:rational}
\subfile{itcs18paper/sec-rational}

\section{Relaxed rationality: HardMax \& Random}
\label{sec:random}
\subfile{itcs18paper/sec-random}

\section{SoftMax response function}
\label{sec:soft}
\subfile{itcs18paper/sec-soft}

%\section{Economic implications}
%\label{sec:welfare}
%\subfile{itcs18paper/sec-welfare}

\gaedit{
\subfile{content/sim_details}
}

\subfile{ec19paper/content/perf_in_iso}

\subfile{ec19paper/content/inverted_u}

\subfile{ec19paper/content/barriers}

\subfile{ec19paper/content/revisited}

\subfile{ec19paper/content/non_greedy_choice}

\gaedit{
\subfile{content/conclusion}
}


\newpage
\bibliographystyle{plain}
\bibliography{bib-abbrv,bib-ML,refs,bib-bandits,bib-AGT,bib-slivkins, bib-random}

\begin{appendices}

In the appendix, we provide background on multi-armed bandits as well as several discussions and proofs omitted from the main text. Furthermore, we provide plots and tables for our experiments, which were omitted from the main text. In all cases, the plots and tables here are in line with those in the main text, and lead to similar qualitative conclusions.

\section{Background on multi-armed bandits}
\label{app:examples}
\subfile{itcs18paper/app-examples}

\section{Non-degeneracy via a random perturbation}
\label{app:perturb}
\subfile{itcs18paper/app-perturb}

\section{Omitted Proofs}
\label{app:omitted_proofs}
\subfile{content/omitted_proofs}

\subfile{ec19paper/content/appendix_for_one_version}

\end{appendices}
\end{document}
%%% Local Variables:
%%% mode: latex
%%% TeX-master: t
%%% End:
