\documentclass[format=acmsmall, review=false]{acmart}

\usepackage{ec19paper/acm-ec-19}
\usepackage{itcs18paper/slivkins-setup,itcs18paper/slivkins-theorems}
\usepackage{tikz} % SW: Added for plots
\usepackage{subfiles}
\usepackage[toc,page]{appendix}
\usepackage{url}            % simple URL typesetting
\usepackage{booktabs}       % professional-quality tables
\usepackage{nicefrac}       % compact symbols for 1/2, etc.
\usepackage{microtype}      % microtypography
\usepackage{amsmath,amssymb,amsfonts}
%\usepackage{natbib} % cannot use with AAAI! A.S.

\usepackage{amsthm} % The amsthm package provides extended theorem environments
\usepackage{float}
\usepackage{sgame, tikz} % Game theory packages
\usepackage{caption}
\usepackage{algorithm,algpseudocode}
\usepackage{makecell}
 \usepackage{multirow}
 \usepackage{graphicx}
\theoremstyle{definition}
%\newtheorem{definition}{Definition}
\newtheorem{finding}{Finding}
%\newtheorem{conjecture}{Conjecture}
\newtheorem{puzzle}{Puzzle}
\captionsetup{font=footnotesize}

% Commands from EC 19 paper
\newcommand{\term}[1]{\ensuremath{\mathtt{#1}}\xspace}
\newcommand{\TS}{\term{TS}}
\newcommand{\DEG}{\term{DEG}}
\newcommand{\DG}{\term{DG}}
\newcommand{\AD}{\term{AD}}
\newcommand{\EGR}{\term{EGR}}
\newcommand{\GR}{\term{GR}}
\newcommand{\Beta}{\term{Beta}} % for Beta distribution
\newcommand{\Eeog}{\term{EoG}} % shorthand for "effective end of game"

\newcommand{\MRV}{mean reward vector\xspace} % mean reward vector
\newcommand{\MRVs}{mean reward vectors\xspace} % mean reward vector

\newcommand{\HMR}{\term{HMR}}
\newcommand{\HM}{\term{HM}}


% commands from ITCS 18 paper

\DeclareMathOperator*{\Expectation}{\mathbb{E}}
\newcommand{\Ex}[2]{\Expectation_{#1}\left[#2\right]}


% notation
%%% advanced notation
\newcommand{\OPT}{\term{OPT}}
\newcommand{\rew}{\term{rew}}  % Bayesian-expected reward after n local rounds
\newcommand{\PMR}{\term{PMR}} % posterior mean reward
\newcommand{\support}{\term{support}}

\newcommand{\BIR}{\term{BIR}} % Bayesian Instantaneous Regret
\newcommand{\regret}{R^{\term{inst}}} % regret
\newcommand{\regretWC}{\regret_{\term{wc}}} % regret

% response function
\newcommand{\respF}{f_{\term{resp}}}
\newcommand{\respEps}{\eps_\term{resp}}

\newcommand{\BReg}{\term{BReg}}
\newcommand{\HardMax}{\term{HardMax}}
\newcommand{\HardMaxRandom}{\term{HardMax\&Random}}
\newcommand{\SoftMaxRandom}{\term{SoftMax}}
\newcommand{\Uniform}{\term{Uniform}}
\newcommand{\Random}{\term{Random}}

\newcommand{\StaticGreedy}{\term{StaticGreedy}}
\newcommand{\DynGreedy}{\term{DynamicGreedy}}
\newcommand{\DynamicGreedy}{\term{DynamicGreedy}}

\newcommand{\termSub}[2]{\ensuremath{\mathtt{#1}_{#2}}\xspace}
\newcommand{\alg}[1][]{\termSub{alg}{#1}}
%\newcommand{\prin}[1][]{\termSub{prin}{#1}}  % principal
\newcommand{\agent}[1][]{\termSub{agent}{#1}}

% priors and posteriors
\newcommand{\prior}{\ensuremath{\mP}\xspace}
\newcommand{\priorMu}{\ensuremath{\prior_\mathtt{mean}}\xspace}
\newcommand{\posteriorN}[2]{\mN_{#1,#2}}  % \posteriorN{principal}{round}

% rationality / innovation / competition
\newcommand{\termTXT}[1]{{\em {#1}}\xspace}

\newcommand{\rationality}{\termTXT{rationality}}
\newcommand{\Rationality}{\termTXT{Rationality}}
\newcommand{\innovation}{\termTXT{innovation}}
\newcommand{\Innovation}{\termTXT{Innovation}}
\newcommand{\competition}{\termTXT{competition}}
\newcommand{\Competition}{\termTXT{Competition}}
\newcommand{\competitiveness}{\termTXT{competitiveness}}
\newcommand{\exploration}{\termTXT{exploration}}
\newcommand{\Exploration}{\termTXT{Exploration}}
% a very useful package for edits and comments, from David Kempe (USC)
\usepackage{ec19paper/color-edits}
%\usepackage[suppress]{color-edits}  % use this to suppress the package
\addauthor{as}{red}    % as for Alex
\addauthor{ga}{blue}  % ga for Guy
\addauthor{sw}{orange} % sw for Steven
% e.g. for Alex, provides \asedit{}, \ascomment{} and \asdelete{}.
\setlength{\tabcolsep}{8pt} % Default value: 6pt
\renewcommand{\arraystretch}{1.5} % Default value: 1

%drt24 hacks
% Letter paper

% Hack to try to make acmart work with biblatex: https://tex.stackexchange.com/questions/37076/is-it-possible-to-load-biblatex-with-a-class-that-has-already-loaded-natbib
\let\citename\relax
%\RequirePackage[abbreviate=true, dateabbrev=true, isbn=true, doi=true, urldate=comp, url=true, maxbibnames=9, backref=false, backend=biber, style=ACM-Reference-Format, language=american]{biblatex}



\usepackage{booktabs} % For formal tables


% Copyright
%\setcopyright{none}
%\setcopyright{acmcopyright}
%\setcopyright{acmlicensed}
\setcopyright{rightsretained}
%\setcopyright{usgov}
%\setcopyright{usgovmixed}
%\setcopyright{cagov}
%\setcopyright{cagovmixed}


\begin{document}
\title[Competing Bandits]
{Competing Bandits}


 \author{Guy Aridor}
 \affiliation{ \institution{Columbia University}}
 \author{Aleksandrs Slivkins}
 \affiliation{\institution{Microsoft Research, New York City}}
 \author{Zhiwei Steven Wu}
 \affiliation{\institution{University of Minnesota - Twin Cities}}
 
 \authorsaddresses{%
  Authors' addresses: Guy Aridor, Columbia University, New York, NY, g.aridor@columbia.edu; Aleksandrs Slivkins, Microsoft Research, New York, NY, slivkins@microsoft.com; Zhiwei Steven Wu, University of Minnesota - Twin Cities, Minneapolis, MN, zsw@umn.edu}

\begin{abstract}
  We empirically study the interplay between \textit{exploration} and
  \textit{competition}. Systems that learn from interactions with
  users often engage in \emph{exploration}: making potentially
  suboptimal decisions in order to acquire new information for future
  decisions. However, when multiple systems are competing for
    the same market of users, exploration may hurt a system's
    reputation in the near term, with adverse competitive effects. In particular, a system may enter a ``death spiral", when the short-term reputation cost decreases
    the number of users for the system to learn from, which degrades the
    system's performance relative to competition and further decreases
    the market share.

We ask whether better exploration algorithms are incentivized under competition. We run extensive numerical experiments in a stylized duopoly model in which two firms deploy multi-armed bandit algorithms and compete for myopic users.  We find that duopoly and monopoly tend to favor a primitive ``greedy algorithm" that does not explore and leads to low consumer welfare, whereas a temporary monopoly (a duopoly with an early entrant) may incentivize better bandit algorithms and lead to higher consumer welfare. Our findings shed light on the first-mover advantage in the digital economy by exploring the role that data can play as a barrier to entry in online markets.
\end{abstract}

%
% The code below should be generated by the tool at
% http://dl.acm.org/ccs.cfm
% Please copy and paste the code instead of the example below.
%
\maketitle
\keywords{Multi-armed bandits,Exploration,Competition,Competition vs Innovation}


\section{Introduction}
\label{sec:intro}
\subfile{itcs18paper/sec-intro}

\section{Related work}
\label{sec:related-work}
\subfile{itcs18paper/sec-related-work}

\section{Our model and preliminaries}
\label{sec:model}
\subfile{itcs18paper/sec-model}

\section{Full rationality (HardMax)}
\label{sec:rational}
\subfile{itcs18paper/sec-rational}

\section{Relaxed rationality: HardMax \& Random}
\label{sec:random}
\subfile{itcs18paper/sec-random}

\section{SoftMax response function}
\label{sec:soft}
\subfile{itcs18paper/sec-soft}

\section{Economic implications}
\label{sec:welfare}
\subfile{itcs18paper/sec-welfare}

\subfile{ec19paper/content/introduction}

\subfile{ec19paper/content/model}

\subfile{ec19paper/content/perf_in_iso}

\subfile{ec19paper/content/inverted_u}

\subfile{ec19paper/content/barriers}

\subfile{ec19paper/content/revisited}

\subfile{ec19paper/content/non_greedy_choice}

\subfile{ec19paper/content/conclusion}

\newpage
\bibliographystyle{acm}
\bibliography{bib-ML,refs,bib-abbrv-short,bib-bandits,bib-AGT,bib-slivkins}

\subfile{ec19paper/content/appendix_for_one_version}

\end{document}
%%% Local Variables:
%%% mode: latex
%%% TeX-master: t
%%% End:
