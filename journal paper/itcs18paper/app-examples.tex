
\newcommand{\ExplorExploit}{\term{ExplorExploit}}
\newcommand{\PhasedExplorExploit}{\term{PhasedExplorExploit}}
%\newcommand{\RandomDynGreedy}{\term{RandomDynGreedy}}
%\newcommand{\SuccesiveElimination}{\term{SuccesiveElimination}}
\newcommand{\SuccesiveEliminationReset}{\term{SuccesiveEliminationReset}}

\newcommand{\IReg}{R^{\term{inst}}} % instantaneous regret

%This appendix provides some pertinent background on multi-armed
%bandits (\emph{MAB}). We discuss \BIR and monotonicity of several MAB algorithms, touching upon: \DynGreedy and \StaticGreedy (Section~\ref{sec:MAB-greedy}), ``naive" MAB algorithms that separate exploration and exploitation (Section~\ref{sec:MAB-naive}), and ``smart" MAB algorithms that combine exploration and exploitation (Section~\ref{sec:MAB-smart}). As we do throughout the paper, we focus on MAB with i.i.d. rewards and a Bayesian prior; we call it \emph{Bayesian MAB} for brevity.


% For a given mean reward vector $\mu$, the $n$-th step instantaneous regret is
%\begin{align}
%\regret(n\mid\mu) &:= \max_{a\in A} \mu_a - \rew(n\mid\mu),\\
%\regretWC(n)    &:=  \sup_{\text{mean reward vectors $\mu$}} \; \BIR(n\mid \mu).
%\end{align}

\ascomment{TODO: insert an intro para.}

Throughout this section, we consider a standard MAB setting. There are $T$ rounds and $K$ arms. In each round $t\in [T]$, the algorithm chooses an arm $a_t\in A$ and receives a reward $r_t\in[0,1]$ for this arm, drawn from a fixed but unknown distribution. The set of all arms is $A$; mean reward of arm $a$ is denoted $\mu_a$. 

Throughout, we let
    $\rew(t) = \mu_{a_t}$
denote the instantaneous mean reward of the algorithm.

Instantaneous regret is denoted
    $\IReg(t) := \max_{a\in A} \mu_a - \mu_{a_t}$,
so that 
    $\BIR(t) = \E[\IReg(t)]$.

\subsection{Monotonicity-in-information for the greedy step}
\label{sec:MAB-greedy}

\ascomment{TODO: insert an intro sentence.}

\ascomment{Steven: NB: we need monotonicity-in-info to argue about "a little greedy goes a long way".}
 
The $n$-step history is the random sequence 
    $H_n = \rbr{ (a_t,r_t):\, t\in[n]}$.
Realizations of $H_n$ are called \emph{realized histories}. 
Let $\mH_n$ be the set of all possible values of $H_n$. 

The \emph{Bayesian-greedy step} given an $n$-step history $h\in \mH_n$ is defined as
\[ \DG(h) := \argmax_{a\in A} \E\sbr{ \mu_a\mid H_n = h },
\quad\text{ties broken uniformly}.\]
 
\begin{lemma}\label{lm:MII}
Let $h,h'$ be two realized histories such that $h$ is a prefix of $h'$. Then
\[ \E\sbr{ \mu_{\DG(h)} } \leq \E\sbr{ \mu_{\DG(h')} }. \]
\end{lemma}
\begin{corollary}
\DynGreedy is monotone.
\end{corollary}

\ascomment{TODO: find a citation for Lemma~\ref{lm:MII}.}

\ascomment{OLD TEXT:}


\begin{lemma}\label{dgmono}
\DynGreedy is monotone, in the sense that $\E[\rew(n)]$ is non-decreasing.
\ascomment{Steven: do we ever need the second sentence here?}
Further, $\E[\rew(n)]$ is strictly increasing for every time step $n$ with $\Pr[a_n\neq a_{n+1}]>0$.
\end{lemma}


\begin{proof}
We prove by induction on $n$ that $\rew(n)\leq \rew(n+1)$ for
\DynGreedy. Let $a_n$ be the random variable recommended at time
$t$, then $\E[\mu_{a_n}| \mI_n ]=\rew(n)$. We can rewrite this as:
\[
\rew(n)=\E_{\mI_n}[\E_{r_n}[\mu_{a_n}|r_n,\mI_n]] =
\E_{\mI_{n+1}}[\mu_{a_n}|\mI_{n+1}]
\]
since $\mI_{n+1}=(\mI_n,r_n)$. At time $n+1$ \DynGreedy will select
an action $a_{n+1}$ such that:
\[
\rew(n+1)=\E[\mu_{a_{n+1}}|\mI_{n+1}]\geq \E[\mu_{a_n}
|\mI_n]=\rew(n)
\]
%
%Now after executing action $a_n$ and observing its reward $r_n$
%there are two possibilities. If $a_n$ is still the best action, then
%its expected reward given the information at time $n$ has not
%changed. On the other hand, if $a_n$ is not the best action anymore,
%then it implies that we recommend a better action. In such a case,
%we increase the expected reward. \ymcomment{need to make this
%formal}
which proves the monotonicity. In cases that $\Pr[a_n\neq
a_{n+1}]>0]$ we have a strict inequality, since with some
probability we select a better action then the realization of $a_n$.
\end{proof}


%%%%%%%%%%%%%%%%%%%%%%%

\subsection{Monotone MAB algorithms}
\label{app:MAB-monotone}

\ascomment{Steven: NB: this is the point of this subsection...}
This subsection argues that ``monotonicity" is a mild assumption in our context. To this end, we present versions of some standard MAB algorithms that are monotone and satisfy ``standard" regret bounds.


\subsubsection{Monotone, exploration-separating MAB algorithms}
\label{app:MAB-monotone-naive}


%\ascomment{Nicer to have $\BIR(n)\leq n^{-1/3}$.
%Explore-then-exploit does not have that! So, let's use phases of
%exponentially increasing duration.}

We provide two exploration-separating algorithms that are monotone and achieve $\IReg(t)=\tildeO\rbr{t^{-1/3}}$. (As we saw in Appendix~\ref{sec:related-classes}, this is the ``right" regret rate for exploration-separating algorithms.)

\ascomment{Steven: can't we prove that Epsilon-Greedy is monotone, too?}

We start with a simple \emph{Explore-First} algorithm: 

\begin{lemma}
Consider an algorithm which initially explores each action $m$ times (where $m$ is a parameter), chooses an action  with the highest observed average reward, and uses it for the remaining steps. In the initial exploration phase, actions are randomly permuted. 
\begin{itemize}
\item[(a)] This algorithm is monotone, for every \MRV.
\ascomment{Steven: right?}

\item[(b)] Fix $\delta>0$ and 
    $m = T^{2/3}\log(|A|/\delta)$.
Then, with probability $1-\delta$, it holds that  
    $\IReg(t)=\tildeO\rbr{T^{-1/3}}$
for each step $n> m\,|A|$.
\ascomment{Steven: can you derive a statement on regret, too?}
\end{itemize}
\end{lemma}

\begin{proof}
In the explore phase we we approximate for each action $a\in A$, the
value of $\mu_a$ by $\hat{\mu}_a$. Using the standard Chernoff
bounds we have that with probability $1-\delta$, for every action
$a\in A$ we have $|\mu_a -\hat{\mu}_a| \leq T^{-1/3}$.

Let $a^* = \arg\max_a \mu_a$ and $a^{ee}$ the action that
\ExplorExploit selects in the explore phase after the first
$|A|T^{2/3}$ agents. Since $\hat{\mu}_{a^*} \leq
\hat{\mu}_{a^{ee}}$, this implies that $\mu_{a^*} -
\mu_{a^{ee}}=O(T^{-1/3})$.

To show that \ExplorExploit$(m)$ is monotone, we need to show only
that $\rew(mK) \leq \rew(mK+1)$. This follows since for any $t< mK$
we have $\rew(t)=\rew(t+1)$, since the recommended action is
uniformly distributed for each time $t$. Also, for any $t\geq mK+1$
we have $\rew(t)=\rew(t+1)$ since we are recommending the same
exploration action. The proof that $\rew(mK) \leq \rew(mK+1)$ is the
same as for \DynGreedy in Lemma~\ref{dgmono}.
\end{proof}

We consider a slightly more advanced version, called \PhasedExplorExploit, which enjoys a similar regret bound uniformly over all steps.

The algorithm is defined as follows. The time is partition in to phases $i = 1,2, \ldots $ of $m_i\geq K$ rounds each, where $m_i$ is chosen in advance and increase with $i$. In each phase $i$, $K$ rounds are chosen at random for exploration, and are assigned randomly to one arm each. The remaining rounds are for exploitation. \ascomment{Steven: exploit how? Bayesian or frequentist? Do we use the data from the same phase?}

\begin{lemma}
Posit $K=2$ arms and Bernoulli rewards. 
\begin{itemize}
\item[(a)] Algorithm \PhasedExplorExploit is monotone, for every \MRV.
    \ascomment{Steven: right?}

\item[(b)] For phase durations $m_i =\sqrt{i}$, it has 
    $\IReg(n)=O(n^{-1/3}+e^{-O(\Delta^2 n^{2/3})}))$,
where $\Delta = |\mu_1-\mu_2|$.
\ascomment{Steven: can you derive $O(T^{2/3})$ regret, too?}
\end{itemize}
\end{lemma}

\begin{proof}
We first show that it is monotone. Recall that $\mu_1>\mu_2$. 
\ascomment{Steven: we only have it for prior means, \ie 
$\E[\mu_1]>\E[\mu_2]$}

Let
$S_i=\sum_{j=1}^t r_{i,j}$ be the sum of the rewards of action $i$
up to phase $t$. We need to show that $\Pr[S_1>S_2]+ (1/2)
\Pr[S_1=S_2]$ is monotonically increasing in $t$. Consider the
random variable $Z=S_1-S_2$. At each phase it increases by $+1$ with
probability $\mu_1(1-\mu_2)$, decreases by $-1$ with probability
$(1-\mu_1)\mu_2$ and otherwise does not change.

Consider the values of $Z$ up to phase $t$. We really care only
about the probability that is shifted from positive to negative and
vice versa.

First, consider the probability that $Z=0$. We can partition it to
$S_1=S_2=r$ events, and let $p(r,r)$ be the probability of this
event. For each such event, we have $p(r,r)\mu_1$ moved to $Z=+1$
and $p(r,r)\mu_2$ moved to $Z=-1$. Since $\mu_1>\mu_2$ we have that
$p(r,r)\mu_1\geq p(r,r)\mu_2$ (note that $p(r,r)$ might be zero, so
we do not have a strict inequality).

Second, consider the probability that $Z=+1$ or $Z=-1$. We can
partition it to $S_1=r+1;S_2=r$ and $S_1=r;S_2=r+1$ events, and let
$p(r+1,r)$ and $p(r,r+1)$ be the probabilities of those events.
%
It is not hard to see that $p(r+1,r)\mu_2=p(r,r+1)\mu_1$.
%
This implies that the probability mass moved from $Z=+1$ to $Z=0$ is
identical to that moved from $Z=-1$ to $Z=0$.

We have showed that $\Pr[S_1>S_2]+ (1/2) \Pr[S_1=S_2]$ and therefore
the expected valued of the exploit action is non-decreasing. Since
we have that the size of the phases are increasing, the $\BIR$ is
strictly increasing between phases and identical within each phase.

We now analyze the $\BIR$ regret. Note that agent $n$ is in phase
$O(n^{2/3})$ and the length of his phase is $O(n^{1/3})$. The $\BIR$
has two parts. The first is due to the exploration, which is at most
$O(n^{-1/3})$. The second is due to the probability that we exploit
the wrong action. This happens with probability $\Pr[S_1<S_2]+ (1/2)
\Pr[S_1=S_2]$ which we can bound using a Chernoff bound by
$e^{-O(\Delta^2n^{2/3})}$, since we explored each action
$O(n^{2/3})$ times.
\end{proof}

\begin{remark}
Actually we have a tradeoff depending on the parameter $m_t$ between
the regret due to exploration and exploitation. (Note that the
monotonicity is always guarantee assuming $m_t$ is monotone.) If we
can set that $m_t = 2^t$ then at time $n$ we have $2/ n$ probability
of an exploit action. For the explore action we are in phase $\log
n$ so the probability of a sub-optimal explore action is
$n^{-O(\Delta^{-2})}$. This should give us
$\BIR(n)=O(n^{-O(\Delta^{-2})})$.
\end{remark}


%MAB algorithm \RandomDynGreedy$(q_t)$for agent $t$ with probability
%$q_t$ selects a random action from $A$ and with probability $1-q_t$
%recommends the best action according to the current posterior
%(similar to \DynGreedy).
%\begin{lemma}
%%Assume that rewards in the range $[0,1]$.
%The algorithm \RandomDynGreedy$(q_t)$, where $q_t=t^{-1/3}$, has,
%with probability $1-\delta$, \BIR$(n)=O(K^{1/2}n^{-1/3}\log
%(K/\delta))$.
%\end{lemma}
%
%\begin{proof}
%At time $n$ we have that with probability $1-\delta/(2K)$, each
%action $a\in A$ was selected at least $n_a=\Theta(n^{2/3}/K)$ times.
%%
%This implies that with probability $1-\delta/(2K)$, for each $a$ we
%have $|\hat{\mu}_a-\mu_a|\leq K^{1/2}n^{-1/3}\log((2K)/\delta)$.
%%
%Therefore, the \BIR is $O(K^{1/2}n^{-1/3}\log((K)/\delta))$
%\end{proof}

%\ascomment{Add: can we modify Phased Explore-then-exploit to make it
%monotone? Perhaps choose exploration rounds u.a.r. within the
%phase?}

%
%\begin{lemma}
%\ExplorExploit$(m)$ is monotone
%\end{lemma}
%
%\begin{proof}
%We need to consider only time $m'=|A|m+1$. For any $n<m'$, we
%clearly have $\rew(n)=\rew(1)$.
%
%Consider two actions $a_1,a_2\in A$, such that $\mu_{a_1} \geq
%\mu_{a_2}$. We claim that the probability that $a_1$ is selected at
%time $m$ to be the exploit action is larger than the probability
%that $a_2$ is selected. This follows since $\hat{\mu}_{a_1}$
%stochastically dominates $\hat{\mu}_{a_2}$, which implies that for
%any threshold $\theta$ we have $\Pr[\hat{\mu}_{a_1}\geq\theta]\geq
%\Pr[\hat{\mu}_{a_2}\geq\theta]$.
%
%After the elimination we consider the expected reward of the
%selected action $\sum_{i\in A} \mu_i q_i$, where $q_i$ is the
%probability that action $i$ was selected for the exploit phase. We
%have that $q_i \geq q_{i+1}$, from the stochastic dominance.
%
%The sum $\sum_{i\in A} \mu_i q_i$ with $q_i \geq q_{i+1}$ and
%$\sum_i q_i=1$ is minimized by setting $q_i=1/|A|$. (We can see that
%if there are $q_i\neq 1/|A|$, then there are two $q_{i}> q_{i+1}$,
%and one can see that setting both to $(q_{i}+ q_{i+1})/2$ decreases
%the value.) Therefore we have that the $\rew(m')\geq\rew(m'-1)$.
%\end{proof}


\subsubsection{Monotone, adaptive-exploration MAB algorithms}
\label{app:MAB-monotone-smart}

We provide a simple modification of Successive Elimination algorithm from \citep{EvenDar-icml06} that is monotone and (still) enjoys the same bounds on regret and BIR. 
%MAB algorithm \SuccesiveElimination holds a set of action
%$A_s\subset A$ which are called {\em surviving actions} and for
%agent $t$ selects a random action from $A_s$. Let $n_{i,t}$ be the
%number of times action $i$ has been selected up to time $t$, and
%$\hat{\mu}_{i,t}$ be the average of the rewards of action $i$ up to
%time $t$ and $\hat{\mu}^*=\max_i \hat{\mu}_{i,t}$. We eliminate
%action $i$ at time $t$, i.e., delete it from $A_s$, if
%$\hat{\mu}_t^*-\hat{\mu}_{i,t} > \log(T/\delta)/\sqrt{n/K}$.
%
%\begin{lemma}
%Assume that the prior is independent and if $\mu_i\geq \mu_j$ then
%$r_i$ stochastically dominates $r_j$.
%%
%The algorithm \SuccesiveElimination, has, with probability
%$1-\delta$, \BIR$(n)=O(\log(T/\delta)/\sqrt{n/K})$.
%\end{lemma}

%\begin{proof}
%Let the best action be $a^*=\arg\max_a \mu_a$. With probability
%$1-\delta$ at any time $n$ we have that for any action $i\in A_s$
%that $|\hat{\mu}_i -\mu_i|\leq \log(T/\delta)/\sqrt{n/K}$, and
%$a^*\in A_s$. This implies that any action $a$ such
%$\mu_{a^*}-\mu_{a}> 3\log(T/\delta)/\sqrt{n/K}$ is eliminated.
%Therefore, any action in $A_s$ has \BIR$(n)$ of at most
%$O(\log(T/\delta)/\sqrt{n/K})$. \ymcomment{should we get the
%constant right?}
%\end{proof}




%\ascomment{Add: Successive Elimination is monotone (or a simple
%modification thereof).}
%
%\begin{lemma}
%\SuccesiveElimination is monotone
%\end{lemma}
%
%\begin{proof}
%We will first show that for any two actions $a_1,a_2\in A$ that if
%$\mu_{a_1} \geq \mu_{a_2}$ then at any time $t$ we have that the
%probability that $a_1$ is eliminated is smaller than the probability
%that $a_2$ is eliminated, i.e., $\Pr[a_1 \not\in A_{s,t+1}| a_1\in
%A_{s,t} ]\geq \Pr[ a_2 \not\in A_{s,t+1}| a_2\in A_{s,t}]$.
%
%We first show that given this statement, the lemma holds and
%\SuccesiveElimination is monotone. Consider $\rew(t)$ versus
%$\rew(t+1)$. If the elimination probabilities would have been
%identical then we would have $\rew(t)=\rew(t+1)$. Consider the top
%two actions, and consider the elimination process in two phases,
%firs we eliminate both with the same probability, and hence the
%reward is unchanged, and then we eliminate only the second, which
%implies that we increase the the reward. The claim about the
%elimination probabilities rules out the case that the elimination
%probability of the best action is higher than then of the the second
%best. Therefore, overall the reward can only increase.
%
%
%We now need to prove the claim. Since $\mu_{a_1}\geq \mu_{a_2}$ and
%they use the same parametric family $\psi_a(\cdot)$ then
%$\hat{\mu}_{a_1}$ stochastically dominates $\hat{\mu}_{a_2}$. We
%actually need to show that $\hat{\mu}_{a_1}$ stochastically
%dominates $\hat{\mu}_{a_2}$ given that $a_1,a_2\in A_{s,t}$. If this
%holds, then the claim about the elimination would be immediate,
%since for any threshold $\theta$ we have $\Pr[\hat{\mu}_{a_1,t} <
%\theta |a_1\in A_{s,t}] \leq \Pr[\hat{\mu}_{a_2,t} < \theta |a_2\in
%A_{s,t}] $. By setting $\theta=\hat{\mu}^*-
%\log(T/\delta)/\sqrt{n/K}$ we derive the lemma.
%
%We now need to prove the claim. We first show that
%$\E[\hat{\mu}_{a_1}-\hat{\mu}_{a_2} | a_1,a_2\in A_s]\geq 0$ which
%is equivalent to $\E[\hat{\mu}_{a_1}-\hat{\mu}_{a_2} |
%\hat{\mu}_{a_1},\hat{\mu}_{a_2}\geq \hat{\mu}^*-\theta]\geq 0$ where
%$\theta$ is the threshold used in \SuccesiveElimination. Since
%action $a_1$ has a high expectation than $a_2$ it holds (at least
%for Bernoulli r.v.).
%
%
%[[OLD]]
%
% Clearly, for any actions $a\in A$ we have that $\Pr[a \in
%A_s]$ decreases with time. In addition, we have that with
%probability $1-\delta$ for every time $t$ we have $a^*\in A_s$.
%
%If $\mu_{a_1} \geq \mu_{a_2}$ then at any time $t$ we have
%$\Pr[a_1\in A_s]\geq \Pr[a_2\in A_s]$.
%
%
%We probably need to show that if $\mu_{a_1} \geq \mu_{a_2}$ then at
%any time $t$ we have that the probability that $a_1$ is eliminated
%is smaller than the probability that $a_2$ is eliminated, i.e.,
%$\Pr[a_1\in A_{s,t}, a_1 \not\in A_{s,t+1}]\geq \Pr[a_2\in A_s, a_2
%\not\in A_{s,t+1}]$.
%\end{proof}




%\ymcomment{Here is a simple modification of Successive-Elimination}



The algorithm, called \SuccesiveEliminationReset, proceeds in phases. In each phase $j$, the algorithm is given a set of ``surviving" arms $A_j\subseteq A$, with $A_1=A$. The phase lasts $|A_j|$ rounds, randomly permuted across the arms in $A_j$ (so that each arm in $A_j$ is chosen exactly once). 

\ascomment{Stopped here.}

\ascomment{Steven: I don't understand the algorithm! It resets after each phase, right? But then, the history is just one sample per each arm? WTF?}

Let $\hat{\mu}_{i,t}$ be the average reward of action $i$ up
to phase $t$ and $\hat{\mu}^*=\max_i \hat{\mu}_{i,t}$. We eliminate
action $i$ at the end of phase $t$, i.e., delete it from $A_s$, if
$\hat{\mu}_t^*-\hat{\mu}_{i,t} > \log(T/\delta)/\sqrt{t}$.
%So far identical to \SuccesiveElimination, the difference is how we
%continue after elimination.
In \SuccesiveEliminationReset we simply reset the algorithm with
$A=A_s-A_{e,t}$, where $A_{e,t}$ is the set of eliminated actions
after phase $t$. Namely, we restart $\hat{\mu}_{i,t}$ and ignore the
old rewards before the elimination.

\begin{lemma}
Consider algorithm \SuccesiveEliminationReset.
\begin{itemize}
\item[(a)]
With probability $1-\delta$, $\IReg(n)=O(\log(T/\delta)/\sqrt{n/K})$.

\item[(b)]
Assuming Bernoulli rewards, the algorithm is monotone.
\ascomment{Steven: for each \MRV, or only in expectation?}
\ascomment{Steven: can you derive optimal regret?}
\end{itemize}
\end{lemma}

\begin{proof}
\xhdr{Part (a).}
Let the best action be $a^*=\arg\max_a \mu_a$. With probability
$1-\delta$ at any time $n$ we have that for any action $i\in A_s$
that $|\hat{\mu}_i -\mu_i|\leq \log(T/\delta)/\sqrt{n/K}$, and
$a^*\in A_s$. This implies that any action $a$ such
$\mu_{a^*}-\mu_{a}> 3\log(T/\delta)/\sqrt{n/K}$ is eliminated.
Therefore, any action in $A_s$ has \BIR$(n)$ of at most
$6\log(T/\delta)/\sqrt{n/K}$.

\xhdr{Part (b).}
Consider the first time $T$ an action is eliminated, and let
$T=\tau$ be a realized value of $T$. Then, clearly for $n<\tau$ we
have $\rew(n)=\rew(1)$ .

Consider two actions $a_1,a_2\in A$, such that $\mu_{a_1} \geq
\mu_{a_2}$. At time $T=\tau$, the probability that  $a_1$ is
eliminated is smaller than the probability that $a_2$ is eliminated.
This follows since $\hat{\mu}_{a_1}$ stochastically dominates
$\hat{\mu}_{a_2}$, which implies that for any threshold $\theta$ we
have $\Pr[\hat{\mu}_{a_1}\geq\theta]\geq
\Pr[\hat{\mu}_{a_2}\geq\theta]$.

After the elimination we consider the expected reward of the
eliminated action $\sum_{i\in A} \mu_i q_i$, where $q_i$ is the
probability that action $i$ was eliminated in time $T=\tau$. We have
that $q_i \leq q_{i+1}$, from the probabilities of elimination.

The sum $\sum_{i\in A} \mu_i q_i$ with $q_i \leq q_{i+1}$ and
$\sum_i q_i=1$ is maximized by setting $q_i=1/|A|$. (We can see that
if there are $q_i\neq 1/|A|$, then there are two $q_{i}< q_{i+1}$,
and one can see that setting both to $(q_{i}+ q_{i+1})/2$ increases
the value.) Therefore we have that the $\rew(\tau)\geq\rew(\tau-1)$.

Now we can continue by induction. For the induction, we can show the
property for {\em any} remaining set of at most $k-1$ actions. The
main issue is that \SuccesiveEliminationReset restarts from scratch,
so we can use induction.
\end{proof}
