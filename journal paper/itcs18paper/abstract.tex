Most modern systems strive to learn from interactions with users, and many engage in \emph{exploration}: making potentially suboptimal choices for the sake of acquiring new information. We initiate a study of the interplay between \emph{exploration and competition}---how such systems balance the exploration for learning and the competition for users. Here the users play three distinct roles: they are customers that generate revenue, they are sources of data for learning, and they are self-interested agents which choose among the competing systems.

In our model, we consider competition between two multi-armed bandit algorithms faced with the same bandit instance. Users arrive one by one and choose among the two algorithms, so that each algorithm makes progress if and only if it is chosen.
%We ask whether better algorithms perform better in such competition,
% and whether the competition incentivizes better algorithms
% and improves social welfare.
We ask whether and to what extent competition incentivizes \asedit{the adoption of better bandit algorithms}. We investigate this issue for several models of user response, as we vary the degree of rationality and competitiveness in the model. \asedit{Our findings are closely related to} the ``competition vs. innovation" relationship, a well-studied theme in economics.


