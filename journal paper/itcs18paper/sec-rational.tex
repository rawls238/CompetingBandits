
In this section, we will consider the version in which
the agents are fully rational, in the sense that their response function is \HardMax. We show that principals are not incentivized to \emph{explore}--- \ie to deviate from \DynGreedy. The core technical result is that if one principal adopts \DynGreedy, then the other principal loses all agents as soon as he deviates.

To make this more precise, let us say that two MAB algorithms \emph{deviate} at (local) step $n$ if there is an action $a\in A$ and \asedit{a set of step-$n$ histories of positive probability such that any history $h$ in this set} is feasible for both algorithms, and under this history the two algorithms choose action $a$ with different probability.

\iffalse
\begin{definition}
  Suppose that an algorithm $\pi$ is not dynamic greedy. Then
  {\em the first exploration} for $\pi$ is the first time step where
  there is a non-zero probability that the recommended action has a
  lower expectation than exploitation option.\sw{define...}
\end{definition}
\fi


\begin{theorem}\label{thm:DG-dominance}
Assume \HardMax response function with fair tie-breaking. Assume that \alg[1] is \DynGreedy, and \alg[2] deviates from \DynGreedy starting from some (local) step $n_0<T$. Then all agents in global rounds $t\geq n_0$ select principal $1$.
\end{theorem}

\begin{corollary}\label{cor:DG-dominance}
The competition game between principals has a unique Nash equilibirium: both principals choose \DynGreedy.
\end{corollary}


\begin{remark}
This corollary holds under a more general model which allows time-discounting: namely, the utility of each principal $i$ in each global round $t$ is $U_{i,t}(r_t)$ if this principal is chosen, and $0$ otherwise, where $U_{i,t}(\cdot)$ is an arbitrary non-decreasing function with $U_{i,t}(0)>0$.
\end{remark}

\subsection{Proof of Theorem~\ref{thm:DG-dominance}}

The proof starts with two auxiliary lemmas: that deviating from \DynGreedy implies a strictly smaller Bayesian-expected reward, and that \HardMax implies a ``sudden-death" property: if one agent chooses principal $1$ with certainty, then so do all subsequent agents do. \asedit{We re-use both lemmas in later sections, so we state them in sufficient generality.}


\begin{lemma}\label{lm:DG-rew}
\asedit{Assume that \alg[1] is \DynGreedy, and \alg[2] deviates from \DynGreedy starting from some (local) step $n_0<T$. Then $\rew_1(n_0)>\rew_2(n_0)$. This holds for any response function $\respF$.}
\end{lemma}

\asedit{Lemma~\ref{lm:DG-rew} does not rely on any particular shape of the response function because it only considers the performance of each algorithm without competition.}

\begin{proof}[Proof of Lemma~\ref{lm:DG-rew}]
Since the two algorithms coincide on the first $n_0-1$ steps, it follows by symmetry that histories $H_{1,n_0}$ and $H_{2,n_0}$ have the same distribution. We use a \emph{coupling argument}: w.l.o.g., we assume the two histories coincide,
    $H_{1,n_0} = H_{2,n_0} = H$.

At local step $n_0$, \DynGreedy chooses an action $a_{1,n_0} = a_{1,n_0}(H)$
which maximizes the posterior mean reward given history $H$: for any realized history $h\in \support(H)$ and any action $a\in A$
\begin{align}\label{eq:lm:DG-rew-1}
 \PMR(a_{1,n_0} \mid H = h) \geq \PMR(a \mid H=h).
\end{align}

\ascomment{Rewrote the rest of the proof to account for positive-prob set of histories.}

By assumption \eqref{eq:assn-distinct}, it follows that 
\begin{align}\label{eq:lm:DG-rew-2}
 \PMR(a_{1,n_0} \mid H = h) > \PMR(a \mid H=h)
 \quad \text{for any $h\in \support(H)$ and $a\neq a_{1,n_0}(h)$.}
\end{align}

Since the two algorithms deviate at step $n_0$, there is a set $S\subset \support(H)$ of step-$n_0$ histories such that $\Pr[S]>0$ and any history $h\in S$ satisfies
    $\Pr[a_{2,n_0}\neq a_{1,n_0} \mid H=h]>0$.
Combining this with \eqref{eq:lm:DG-rew-2}, we deduce that 
\begin{align}\label{eq:lm:DG-rew-3}
 \PMR(a_{1,n_0} \mid H = h) > \E\left[ \mu_{a_{2,n_0}}\mid H=h \right]
 \quad\text{for each history $h\in S$}.
\end{align}
Using \eqref{eq:lm:DG-rew-1} and \eqref{eq:lm:DG-rew-3} and integrating over realized histories $h$, we obtain
    $\rew_1(n_0) > \rew_2(n_0)$.
\end{proof}


\begin{lemma}\label{lm:DG-sudden}
\asedit{Consider \HardMax response function with $\respF(0)\geq\tfrac12$.} 
Suppose \alg[1] is monotone, and $\PMR_1(t_0)>\PMR_2(t_0)$ for some global round $t_0$. Then $\PMR_1(t)>\PMR_2(t)$ for all subsequent rounds $t$.
\end{lemma}

\begin{proof}
Let us use induction on round $t\geq t_0$, with the base case $t=t_0$. Let $\mN=\mN_{1,t_0}$ be the agents' posterior distribution for $n_{1,t_0}$, the number of global rounds before $t_0$ in which principal $1$ is chosen. By induction, all agents from $t_0$ to $t-1$ chose principal $1$, so $\PMR_2(t_0)= \PMR_2(t)$. Therefore,
\[ \PMR_1(t)
    = \Ex{n\sim \mN}{\rew_1(n+1+t-t_0)}
    \geq \Ex{n\sim \mN}{\rew_1(n+1)}
    =\PMR_1(t_0) > \PMR_2(t_0)= \PMR_2(t), \]
where the first inequality holds because \alg[1] is monotone, and the second one is the base case.
\end{proof}

\begin{proof}[Proof of Theorem~\ref{thm:DG-dominance}]
Since the two algorithms coincide on the first $n_0-1$ steps, it follows by symmetry that $\rew_1(n) = \rew_2(n)$ for any $n< n_0$.
By Lemma~\ref{lm:DG-rew},
    $\rew_1(n_0) > \rew_2(n_0)$.

Recall that $n_i(t)$ is the number of global rounds $s<t$ in which principal $i$ is chosen, and $\posteriorN{i}{t}$ is the agents' posterior distribution for this quantity. By symmetry, each agent $t<n_0$ chooses a principal uniformly at random. It follows that
    $\posteriorN{1}{n_0} = \posteriorN{2}{n_0}$
(denote both distributions by $\mN$ for brevity), and $\mN(n_0-1)>0$.
Therefore:
\begin{align}
\PMR_1(n_0)
  &= \Ex{n\sim \mN} {\rew_1(n+ 1)}
   = \sum_{n = 0}^{n_0-1} \mN(n) \cdot{\rew_1(n + 1)} \nonumber \\
  & > \mN(n_0-1)\cdot {\rew_2(n_0)} + \sum_{n = 0}^{n_0-2}  \mN(n)\cdot{\rew_2(n + 1)}
    \nonumber \\
  &= \Ex{n\sim \mN}{\rew_2(n + 1)} = \PMR_2(n_0) \label{eq:pf:thm:DG-dominance}
\end{align}
So, agent $n_0$ chooses principal $1$. By Lemma~\ref{lm:DG-sudden} \asedit{(noting that \DynGreedy is monotone)}, all subsequent agents choose principal $1$, too.
\end{proof}


%\begin{theorem}
%  Suppose that the agents break ties uniformly at random, that is
%  $q=1/2$, then both principals playing \DynGreedy is the unique
%  equilibrium.
%\end{theorem}
%
%\begin{proof}
%  Suppose that both principals play dynamic greedy. The result
%  of Theorem~\ref{thm:DG-dominance} shows that there is no beneficial deviation
%  for both principals, so this forms a pure strategy equilibrium.
%
%  Now we will show that no other strategy profile can form equilibrium
%  in this game.  Suppose that both principals play some mixed
%  strategies that places non-zero probability on algorithms other than
%  dynamic greedy. We know one of the principals $p_j$ is getting no
%  more than $T/2$ agents in expectation. If $p_j$ switch to playing
%  only dynamic greedy, he can get strictly more than $T/2$ agents
%  by Theorem~\ref{thm:DG-dominance}. Thus, such a strategy profile will not form
%  an equilibrium.  Finally, suppose that $p_1$ plays dynamic greedy
%  and $p_2$ plays algorithms other than dynamic greedy with non-zero
%  probability. By Theorem~\ref{thm:DG-dominance}, we know that $p_2$ gets less
%  than $T/2$ agents in expectation. This means switching to dynamic
%  greedy will strictly improve the expected utility to $T/2$, so such
%  strategy profile cannot form an equilibrium either.
%\end{proof}


\subsection{\HardMax with biased tie-breaking}
\label{sec:HardMax-biased}

The \HardMax model is very sensitive to the tie-breaking rule. For starters, if ties are  broken deterministically in favor of principal $1$, then principal 1 can get all agents no matter what the other principal does, simply by using \StaticGreedy.

\begin{theorem}\label{thm:HardMax-hardTies}
Assume \HardMax response function with $\respF(0)=1$ (ties are always broken in favor of principal $1$). If \alg[1] is \StaticGreedy, then all agents choose principal $1$.
\end{theorem}

\begin{proof}
Agent $1$ chooses principal $1$ because of the tie-breaking rule. Since \StaticGreedy is trivially monotone, all the subsequent agents choose principal $1$ by an induction argument similar to the one in the proof of Lemma~\ref{lm:DG-sudden}.
\end{proof}

A more challenging scenario is when the tie-breaking is biased in favor of principal 1, but not deterministically so: $\respF(0)>\tfrac12$. Then this principal also has a ``winning strategy" no matter what the other principal does. Specifically, principal 1 can get all but the first few agents, under a mild technical assumption that \DynGreedy deviates from \StaticGreedy. Principal 1 can use \DynGreedy, or any other monotone MAB algorithm that coincides with \DynGreedy in the first few steps. We state the result and defer the proof to the appendix.

%We can generalize the theorem top the case of $q >1/2$ if the
%principal $p_1$ can guarantee better than the a priori best action to
%all the agents following the second.


\begin{theorem}\label{thm:HardMax-biased}
Assume \HardMax response function with $\respF(0)>\tfrac12$ (\ie tie-breaking is biased in favor of principal $1$). Assume the prior $\mP$ is such that \DynGreedy deviates from \StaticGreedy starting from some step $n_0$. Suppose that principal $1$ runs a monotone MAB algorithm that coincides with \DynGreedy in the first $n_0$ steps. Then all agents $t\geq n_0$ choose principal $1$.
\end{theorem}


% OLD PROOF (assuming alg1 coincides with DG on first two rounds, etc.
% We consider two cases, depending on the arm $a_{2,1}$ that principal $2$ chooses in the same step. The interesting case is $a_{2,1}=1$.
%Then agent $1$ selects principal $1$ with probability $q:=\respF(0)>\tfrac12$. Therefore, agent $2$ forms the following posterior mean rewards:
%\begin{align*}
%  \PMR_1(2) &= (1-q)\cdot \E[\mu_1] + q\cdot \rew_1(2)
%    \EqComment{for principal 1}\\
%  \PMR_2(2) &= q\cdot \E[\mu_1] + (1-q)\cdot \rew_2(2)
%    \EqComment{for principal 2}.
%\end{align*}
%We know that
%    $\rew_1(2) > \E[\mu_1]$
%(\DynGreedy is strictly monotone in the first $2$ rounds)
%from Lemma~\ref{dgmono}, and
%    $\rew_1(2) \geq \rew_2(2)$
%by Lemma~\ref{lm:DG-rew}. It follows that
%    $\PMR_1(2)>\PMR_2(2)$,
%and consequently agent $2$ chooses principal $1$. The subsequent agents choose principal $1$ by Lemma~\ref{lm:DG-sudden}.
%
%The remaining case is $a_{2,1}\neq 1$: principal $p_2$ does not recommend arm $1$ in the first step. Since all other arms have a strictly lower prior mean reward, agent $1$ chooses principal $1$. The subsequent agents choose principal $1$ by Lemma~\ref{lm:DG-sudden}.



% \begin{lemma}
%   Let $\mu^2_{DG}$ the expected reward of action of \DynGreedy
%   in the second step. Then $\mu^2_{DG} > \mu_1$.
% \end{lemma}

% \begin{proof}
%   \sw{todo. This requires an assumption that the second arm has
%     non-zero probability of beating the first one.}
% \end{proof}



% Steven's proof for a more general version with alg1 = BIC algo
%Agent $1$ chooses principal $1$ because of the tie-breaking rule,
%  Since principal $p_1$ in the first step recommends arm 1, the agent
%  selects principal $p_1$ (due to the tie breaking rule).  Agent $2$
%  knows that agent $1$ selected $p_1$. Because of BIC, she knows $p_1$
%  will offer reward no lower than $p_2$ in expectation.  By induction,
%  all the agents select principal $p_1$.



%\sw{We also have an observation saying if one principal starts with a
%  better dynamic greedy will just dominate the game. It could do some
%  amount of exploration while having better Bayesian per-round regret
%  than the other principal. This means the other principal will never
%  have any agents arriving for learning.}

%%% Local Variables:
%%% TeX-master: "main.tex"
%%% End: 