\documentclass[../competing_bandits.tex]{subfiles}
\begin{document}

\section{Non-deterministic Choice Models}\label{sec:non_greedy}


In this section we consider an extension of our base model where the agents' choice model is no longer deterministic. Recall that in our base model agents deterministically choose the firm with the higher reputation score. We define this choice model as \textit{HardMax} (\HM). We study a small deviation from this choice model where the agent now selects between the firms uniformly with probability $0 < \epsilon < 1$ and takes the firm with the higher reputation score with probability $1 - \epsilon$. Following \cite{mansour2018competing} we define this choice model as \textit{HardMax with random} (\HMR).

In Section \ref{sec:competition} we show that in the permanent duopoly case exploration can lead to a death spiral which eventually starved the firm of agents. However, giving one firm a small head start or enough free agents via the warm start incentivizes it to play \TS since it could recover the reputation costs it incurred from early exploration. An interpretation of \HMR~ is that instead of concentrating the free agents as arriving in the beginning of the game instead they are dispersed throughout the game.

\footnotesize
\begin{table*}[t]
\centering
\begin{tabular}{|c|c|c|c||c|c|c|}
  \hline
  & \multicolumn{3}{c||}{Heavy-Tail (\HMR)}
  & \multicolumn{3}{c|}{Heavy-Tail (\HM)}\\
  \hline
  & \TS vs \DG & \TS vs \DEG  & \DG vs \DEG 
 & \TS vs \DG & \TS vs \DEG  & \DG vs \DEG  \\
  \hline
$T = 2000$
 & \makecell{ \textbf{0.43} $\pm$ 0.02 \\Var: 0.15 } 
  & \makecell{ \textbf{0.44} $\pm$ 0.02 \\Var: 0.15 } 
  & \makecell{ \textbf{0.6} $\pm$ 0.02 \\Var: 0.1 }
  %%
 &  \makecell{ \textbf{0.29} $\pm$ 0.03 \\Var: 0.2 } 
  & \makecell{ \textbf{0.28} $\pm$ 0.03 \\Var: 0.19 } 
  & \makecell{ \textbf{0.63} $\pm$ 0.03 \\Var: 0.18 }
    \\
\hline
  $T= 5000$ 
   & \makecell{ \textbf{0.66} $\pm$ 0.01 \\Var: 0.056 } 
  & \makecell{ \textbf{0.59} $\pm$ 0.02 \\Var: 0.092 } 
  & \makecell{ \textbf{0.56} $\pm$ 0.02 \\Var: 0.098 } 
  %%
 & \makecell{ \textbf{0.29} $\pm$ 0.03 \\Var: 0.2 } 
 & \makecell{ \textbf{0.29} $\pm$ 0.03 \\Var: 0.2 } 
 & \makecell{ \textbf{0.62} $\pm$ 0.03 \\Var: 0.19 }
 \\ 
  \hline
  $T = 10000$
  & \makecell{ \textbf{0.76} $\pm$ 0.01 \\Var: 0.026 } 
 & \makecell{ \textbf{0.67} $\pm$ 0.02 \\Var: 0.067 } 
 & \makecell{ \textbf{0.52} $\pm$ 0.02 \\Var: 0.11 }
 %%
  & \makecell{ \textbf{0.3} $\pm$ 0.03 \\Var: 0.21 } 
  & \makecell{ \textbf{0.3} $\pm$ 0.03 \\Var: 0.2 } 
  & \makecell{ \textbf{0.6} $\pm$ 0.03 \\Var: 0.2 }
  %%
  \\
   \hline
\end{tabular}
\normalsize
\caption{Performance Comparison between \HM and \HMR Choice on the Heavy-Tail MAB instance. Each cell describes the market shares in a game between two algorithms, call them Alg1 vs. Alg2, at a particular value of $t$. Line 1 in the cell is the market share of Alg 1: the average (in bold) and the 95\% confidence band.
%For example, the cell in the top left indicates that TS gets on average 64\% of the market when played against DG.
Line 2 specifies the variance of the market shares across the simulations. The results reported here are with $\epsilon = 0.1$ in the \HMR choice model and $T_0 = 20$.}
\label{tab:non_greedy_table}
\end{table*}

\normalsize


\begin{finding}\label{find:non_greedy_choice}
\textit{Under the \HMR~ choice model \TS is weakly dominant as long as $T$ is sufficiently large. \swdelete{However, both $T$ and $\epsilon$ have to be relatively large for this to occur.}}
\end{finding}



Table \ref{tab:non_greedy_table} shows the average market shares under the \HM~ vs \HMR~ choice model. As $T$ gets sufficiently large, \TS becomes weakly dominant for the \HMR model whereas, as expected, it does not for the HardMax model. These results hold across all problem instances.\footnote{The results here are pulled using different \MRV and realizations than the results pulled previously (due to the larger $T$). However, they are drawn from the same prior instances and so qualitatively are the same but the quantitative results are not directly comparable to those from the previous section.} The intuition for why \TS should become the dominant strategy eventually is simple. The existence of the random agents ensures that no algorithm ever gets completely starved of agents. As a result, each algorithm should eventually converge to its asymptotic performance in isolation.

However, it takes a significant amount of randomness and a relatively large time horizon for this convergence to occur. Even with $T = 10000$ and $\epsilon = 0.1$ we see that \DEG still outperforms \DG on the Heavy-Tail MAB instance as well as that \TS only starts to become weakly dominant at $T = 10000$ for the Uniform MAB instance. Table \ref{tab:non_greedy_table} also shows that another difference between the two choice models is that \HMR leads to lower variance in market shares across simulations compared to \HM.

Finding \ref{find:non_greedy_choice} implies that we can re-interpret the inverted-U findings from before in terms of the number of agents that a firm receives without having to worry about incentives. In the extreme when the firm gets all agents for free as in the monopoly case then it is incentivized to play \DG. When it only gets some of the agents for free, either via a large warm start, a temporary monopoly, or non-deterministic choice, then \TS is incentivized. However, if the number of free agents gets small enough then \DG is incentivized as in the permanent duopoly analysis from before.



\end{document}
%%% Local Variables:
%%% mode: latex
%%% TeX-master: "../competing_bandits"
%%% End: 