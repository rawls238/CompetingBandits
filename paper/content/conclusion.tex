\documentclass[../competing_bandits.tex]{subfiles}
\begin{document}

\section{Conclusion}\label{sec:conclusion}

\gaedit{In this paper we considered a setting where firms simultaneously learn from and compete for users. We showed that competition may not always induce firms to commit to algorithms that engage in exploration resulting in welfare losses for consumers. The primary mechanism behind this was that exploration may have short-term reputational consequences that lead to naive algorithms winning in competition that are bad in the long-run. Allowing one firm to have a head-start, or a first-mover advantage, leads to the first-mover having an incentive to play traditionally ``good" bandit algorithms and having significantly better welfare for consumers. Finally, we attempted to isolate the effect that the data advantage of an incumbent has on its ability to garner a considerable market share. Contrary to previous studies and discussions which do not consider competition dynamics or endogenous data collection, we find that the data advantage is substantial in generating significant market power. Additionally we show that this effect is strongest when an incumbent performs exploration in the incumbency period and generates a high-quality dataset on actions that may be difficult for an entrant to generate due to the pressure of competition.}

\end{document}
%%% Local Variables:
%%% mode: latex
%%% TeX-master: "../competing_bandits"
%%% End:
