\documentclass[../competing_bandits.tex]{subfiles}
\begin{document}

\section{Conclusion}\label{sec:conclusion}

We consider a stylized duopoly setting where firms simultaneously learn from and compete for users. We showed that competition may not always induce firms to commit to better exploration algorithms, resulting in welfare losses for consumers. The primary reason is that exploration may have short-term reputational consequences that lead to more naive algorithms winning in a long-term competition. Allowing one firm to have a head start, a.k.a. the first-mover advantage, incentivizes the first-mover to deploy ``better" algorithms, which in turn leads to better welfare for consumers. Finally, we isolate the component of the first-mover advantage that is due to having more initial data, and find that even a small amount of this ``data advantage" leads to substantial long-term market power.

\end{document}
%%% Local Variables:
%%% mode: latex
%%% TeX-master: "../competing_bandits"
%%% End:
