\documentclass[../competing_bandits.tex]{subfiles}
\begin{document}

\section{Data and Reputation as Barriers to Entry}\label{section:6}

An alternative interpretation of the results in the previous section is that the "temporary monopoly" provides a first mover advantage to the incumbent firm and that this first mover advantage allows the firm to both get a data advantage as well as a reputational advantage over the entrant. This provides an alternative interpretation of the classic "first mover advantage" that has been well-studied in economics and marketing \cite{kerin1992first} where, in our model, an incumbent can use data and reputation as a barrier to entry simply due to the fact that they were in the market before the entrant. Which plays a bigger role in preventing the entrant from being able to establish market share? We run two additional experiments, modifying the previous temporary monopoly experiment so that in one experiment the reputation of the incumbent is artificially erased (Reputation Erased Experiment) and another in which the information gained by the incumbent is artificially erased so that the posterior is reset to the prior (Information Erased Experiment).

\begin{table}[ht]
\centering
\caption{Reputation Erased Experiment, Heavy Tail}
\begin{tabular}{c@{\hspace{1.0\tabcolsep}}ccc}
  \hline
 & TS & DEG &  DG \\
  \hline
TS & \makecell{\textbf{ 0.016 } $\pm$ 0.0075 \\Var: 0.01 \\ ES: 100 \%} & \makecell{\textbf{ 0.13 } $\pm$ 0.02 \\Var: 0.1 \\ ES: 97 \%} & \makecell{\textbf{ 0.2 } $\pm$ 0.024 \\Var: 0.1 \\ ES: 96 \%} \\
  DEG & \makecell{\textbf{ 0.068 } $\pm$ 0.013 \\Var: 0.05 \\ ES: 93 \%} & \makecell{\textbf{ 0.29 } $\pm$ 0.024 \\Var: 0.2 \\ ES: 75 \%} & \makecell{\textbf{ 0.26 } $\pm$ 0.024 \\Var: 0.2 \\ ES: 80 \%} \\
   DG & \makecell{\textbf{ 0.15 } $\pm$ 0.019 \\Var: 0.1 \\ ES: 87 \%} & \makecell{\textbf{ 0.38 } $\pm$ 0.028 \\Var: 0.2 \\ ES: 80 \%} & \makecell{\textbf{ 0.33 } $\pm$ 0.024 \\Var: 0.2 \\ ES: 67 \%} \\
   \hline
   \label{rep_erase}
\end{tabular}
\end{table}
% latex table generated in R 3.4.0 by xtable 1.8-2 package
% Sun Jul 22 14:23:06 2018
\begin{table}[ht]
\centering
\caption{Information Erased Experiment, Heavy Tail}
\begin{tabular}{c@{\hspace{1.0\tabcolsep}}ccc}
  \hline
 & TS & DEG &  DG \\
  \hline
TS & \makecell{\textbf{ 0.024 } $\pm$ 0.0094 \\Var: 0.02 \\ ES: 100 \%} & \makecell{\textbf{ 0.16 } $\pm$ 0.022 \\Var: 0.1 \\ ES: 97 \%} & \makecell{\textbf{ 0.22 } $\pm$ 0.025 \\Var: 0.2 \\ ES: 95 \%} \\
  DEG & \makecell{\textbf{ 0.24 } $\pm$ 0.025 \\Var: 0.2 \\ ES: 94 \%} & \makecell{\textbf{ 0.29 } $\pm$ 0.024 \\Var: 0.1 \\ ES: 72 \%} & \makecell{\textbf{ 0.27 } $\pm$ 0.024 \\Var: 0.1 \\ ES: 76 \%} \\
   DG & \makecell{\textbf{ 0.33 } $\pm$ 0.028 \\Var: 0.2 \\ ES: 94 \%} & \makecell{\textbf{ 0.38 } $\pm$ 0.026 \\Var: 0.2 \\ ES: 74 \%} & \makecell{\textbf{ 0.33 } $\pm$ 0.023 \\Var: 0.1 \\ ES: 58 \%} \\
   \hline
   \label{info_erase}
\end{tabular}
\end{table}

\begin{finding}
\textit{Both the data advantage and reputation advantage alone serve as strong barriers to entry. When the incumbent plays $\TS$, the data advantage serves as a larger barrier to entry compared to reputation, but when the incumbent plays $\DEG$ or $DG$ there is no substantial difference between the two.}
\end{finding}

Tables \ref{rep_erase} and \ref{info_erase} display the results of these experiments. Maintaining either the data or reputation advantage alone still allows the incumbent to retain a significant portion of the market. The main interesting finding here, which is robust across priors for this parameterization, is that information serves as a larger barrier to entry than reputation when the incumbent plays $\TS$. One possible intuition for this is that, for Heavy Tail, the information acquired after the $X$ rounds under $\TS$ is sufficient so that it becomes relatively easier for $\TS$ to recover the reputation loss compared to the information loss, but since $\DEG$ and $DG$ learn slower and they've acquired less information over the $X$ rounds relative to $\TS$ there is less of an informational advantage.

Though the setup is purely experimental, it is nonetheless interesting to look at if the same strategies remain best responses in this setting compared to the setting where both data and reputation are retained. We find that the best responses remain the same for the Heavy Tail prior, but for the Uniform prior they change where $\TS$ is weakly dominant for the incumbent under the reputation erased treatment but not under the information erased treatment.

\end{document}