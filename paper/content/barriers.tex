\documentclass[../competing_bandits.tex]{subfiles}
\begin{document}

\section{Data and Reputation as Barriers to Entry}\label{section:6}
In the temporary monopoly setting, the incumbent firm enjoys a first
mover advantage, which allows the firm to acquire data on potentially
suboptimal actions without incurring an immediate reputational
consequence, and to build up a high reputation before the entrant
enters the market. Thus, the early entry gives the incumbent both a
\textit{data} advantage and a \textit{reputational} advantage over the
entrant. This provides a quantitative perspective on the classic
``first mover advantage" \cite{kerin1992first} in the digital economy
where data can serve as a barrier to entry in many markets.

While the two advantages together serve as strong barriers to entry, a
natural question to ask is whether either one of the advantages alone
suffices as a strong barrier to entry. If both of them are barriers of
entry, which of the two is the stronger?

To answer these questions, we isolate the effect from each of the two
advantages and run two additional experiments. In the first experiment
(called \emph{reputation erased experiment}), the reputation of the
incumbent is artificially reset upon the entry of the entrant {so that
  the incumbent's reputation score is only based on the rewards
  generated in the warm start period}. In the second experiment
(called \emph{information erased experiment}), the information gained
by the incumbent is erased upon the entry of the entrant in the sense
that its posterior is reset to the initial prior. The results of the
experiments are presented in Table~\ref{barrier_exp}.


\begin{table*}[h]
\centering
\begin{tabular}{|c|c|c|c||c|c|c|}
\hline
  & \multicolumn{3}{c|}{Information erased}
  & \multicolumn{3}{c|}{Reputation erased}\\
\hline
& $\TS$  & $\DEG$  & $\DG$ 
& $\TS$  & $\DEG$  & $\DG$
\\\hline
$\TS$ 
    & \makecell{\textbf{0.021}$\pm$0.009}
    & \makecell{\textbf{0.16}$\pm$0.02} 
    & \makecell{\textbf{0.21} $\pm$0.02} 
    %%
    & \makecell{\textbf{0.0096}$\pm$0.006}
    & \makecell{\textbf{0.11}$\pm$0.02}
    & \makecell{\textbf{0.18}$\pm$0.02} 
    \\ \hline
$\DEG$ 
    & \makecell{\textbf{0.26}$\pm$0.03} 
    & \makecell{\textbf{0.3}$\pm$0.02} 
    & \makecell{\textbf{0.26}$\pm$0.02} 
    %%
    & \makecell{\textbf{0.073}$\pm$0.01}
    & \makecell{\textbf{0.29}$\pm$0.02}
    & \makecell{\textbf{0.25}$\pm$0.02} 
    \\ \hline
$\DG$ 
    & \makecell{\textbf{0.34}$\pm$0.03} 
    & \makecell{\textbf{0.4}$\pm$0.03} 
    & \makecell{\textbf{0.33}$\pm$0.02} 
    %%
    & \makecell{\textbf{0.15}$\pm$0.02}
    & \makecell{\textbf{0.39}$\pm$0.03}
    & \makecell{\textbf{0.33}$\pm$0.02}
    \\\hline
\end{tabular}
\caption{Information-erased vs. reputation-erased experiment, on Heavy-Tail MAB instance. Each cell describes the duopoly game between the entrant's algorithm (the {\bf row}) and the incumbent's algorithm (the {\bf column}). The cell specifies the entrant's market share for the rounds in which hit was present: the average (in bold) and the 95\% confidence interval. NB: smaller average is better for the incumbent.}
\label{barrier_exp}
\end{table*}

%\begin{table}[ht]
%\centering
%\begin{tabular}{cc|c|c|c|}
%  & \multicolumn{1}{c}{} & \multicolumn{3}{c}{Incumbent} \\
%  & \multicolumn{1}{c}{} & \multicolumn{1}{c}{$\TS$}  & \multicolumn{1}{c}{$\DEG$}  & \multicolumn{1}{c}{$\DG$} \\\cline{3-5}
%            & $\TS$ & \makecell{\textbf{0.021} \\$\pm$0.009}& 		\makecell{\textbf{0.16} \\$\pm$0.02} &
%            \makecell{\textbf{0.21} \\ $\pm$0.02} \\ \cline{3-5}
%Entrant  & $\DEG$ & \makecell{\textbf{0.26} \\$\pm$0.03} & \makecell{\textbf{0.3} \\$\pm$0.02} &
%\makecell{\textbf{0.26} \\ $\pm$0.02} \\ \cline{3-5}
%            & $\DG$ & \makecell{\textbf{0.34} \\ $\pm$0.03} & \makecell{\textbf{0.4} \\$\pm$0.03} &
%            \makecell{\textbf{0.33} \\$\pm$0.02} \\\cline{3-5}
%\end{tabular}
%\caption{Information Erased Experiment Heavy Tail}
%\label{info_erase}
%\end{table}
%
%\begin{table}[ht]
%\centering
%\begin{tabular}{cc|c|c|c|}
%  & \multicolumn{1}{c}{} & \multicolumn{3}{c}{Incumbent} \\
%  & \multicolumn{1}{c}{} & \multicolumn{1}{c}{$\TS$}  & \multicolumn{1}{c}{$\DEG$}  & \multicolumn{1}{c}{$\DG$} \\\cline{3-5}
%            & $\TS$ & \makecell{\textbf{0.0096} \\$\pm$0.006}& 		\makecell{\textbf{0.11} \\$\pm$0.02} &
%            \makecell{\textbf{0.18} \\$\pm$0.02} \\ \cline{3-5}
%Entrant  & $\DEG$ & \makecell{\textbf{0.073} \\ $\pm$0.01} & \makecell{\textbf{0.29} \\ $\pm$0.02} &
%\makecell{\textbf{0.25} \\$\pm$0.02} \\ \cline{3-5}
%            & $\DG$ &\makecell{\textbf{0.15} \\$\pm$0.02} & \makecell{\textbf{0.39} \\$\pm$0.03} &
%           \makecell{\textbf{0.33} \\ $\pm$0.02} \\\cline{3-5}
%\end{tabular}
%\caption{Reputation Erased Experiment Heavy Tail}
%\label{rep_erase}
%\end{table}

\begin{finding}
\swedit{During the rounds in which the entrant is present, the incumbent
  with either data or reputational advantage alone gets a larger share
  than the share of an incumbent without any advantage.  When the
  incumbent plays $\TS$, the data advantage serves as a larger barrier
  to entry compared to reputation, but when the incumbent plays $\DEG$
  or $\DG$ there is no substantial difference between the two
  barriers.}
\OMIT{\textit{Maintaining either the data or reputation advantage alone still allows the incumbent \gaedit{to retain a larger market share compared to the share they get under permanent duopoly}.\\ \indent When the incumbent plays $\TS$, the data advantage serves as a larger barrier to entry compared to reputation, but when the incumbent plays $\DEG$ or $\DG$ there is no substantial difference between the two barriers.}}
\end{finding}

\swcomment{for the first part of the finding, is it true for any algorithm played by the incumbent?}


The reason for the finding is that, under $\TS$,\swcomment{also true
  for other algos?} the early entry allows the incumbent to do
exploration without immediate reputational consequence in the early
rounds and so, even when reputation is erased, the incumbent has
already acquired sufficient information that simply exploiting that
information allows her to quickly recover her
reputation.\swcomment{which part are you explaining?} However, while
the reputational advantage gives the incumbent some cushion for doing
exploration the incumbent is still vulnerable to getting unlucky while
doing exploration and suffering the reputational consequences. This
only happens under $\TS$ since $\TS$, by construction, is more
intelligent about the data that it acquires compared to $\DEG$ and
$\DG$.

\OMIT{Though the setup is purely experimental, it is nonetheless interesting to look at if the same strategies remain best responses in this setting compared to the setting where both data and reputation are retained. We find that the best responses remain the same for the Heavy Tail instance, but for the Uniform instance they change where $\TS$ is weakly dominant for the incumbent under the reputation erased treatment but not under the information erased treatment.}

\end{document}
%%% Local Variables:
%%% mode: latex
%%% TeX-master: "../competing_bandits"
%%% End:
