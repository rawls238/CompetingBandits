\documentclass[../competing_bandits.tex]{subfiles}
\begin{document}
\section{Competition vs Better Algorithms: Inverted-U}\label{section:5}

Now we demonstrate the inverted-U relationship between the
incentivized algorithms and the competition levels. In particular, we
investigate the incentivized algorithms under three different regimes:
permanent monopoly, temporary monopoly, and permanent duopoly:
  
\begin{finding}
\textit{For sufficiently low warm start under the instances that are relative reputation costly, the unique pure Nash equilibrium strategies in the competition game are:\footnote{the uniqueness comes from the fact that we break indifferences towards easier to deploy algorithms}
\begin{center}
\textbf{Permanent Monopoly} - $\DG$ \\
\textbf{Temporary Monopoly} (incumbent)- $\TS$ \\
\textbf{Permanent Duopoly} - $\DG$ \\
\end{center}
The conditions on $\TS$ being dominant under the temporary monopoly are that the incumbent is a temporary monopoly for sufficiently many periods.
}
\end{finding}

\xhdr{Permanent Monopoly.} Since there is only a single firm in the
market for the entire period, the firm can take the entire market
regardless of what algorithm it deploys. \swedit{Since we assume that exploration algorithms ($\DEG$ and $\TS$) would incur deployment cost and firms break indifferences towards easier to deploy algorithms}, the firm would choose to deploy $\DG$.


Next, to describe our results for temporary monopoly, we will
introduce the notion of \textit{effective end of game (EEOG)} --- the
last round $t$ such that the agent at round $(t-1)$ and the next agent
around at $t$ choose different firms.

\xhdr{Permanent Duopoly} \swedit{Since both firms enter at the same time, they will both get an expected market share of $1/2$ whenever they deploy
  the same algorithms. We will mainly investigate the market shares
  when the firms play different algorithms.  In Tables \ref{sim_ht}
  and \ref{sim_nih}, we report the average market share taken over the
  $N$ simulations for different values of warm-up length $T_0$, and
  also the mean and median of the EEOG, under heavy tail and needle in
  haystack instances.}

For the relative reputation costly instances,\footnote{We defer the table for uniform instances to the appendix. Summarizing the results, we see that, for low warm start, \DG yields more than half the market against \TS but is indifferent between \DG and \DEG. Since we break indifference towards easier to deploy algorithms, we also find that \DG is the incentivized algorithm in equilibrium} we see that, for low warm start, $\DG$ is a weakly dominant strategy for the competition game, and in particular weakly dominates $\TS$. This establishes our claim that $\DG$ is the incentivized algorithm. 

The reason we see that $\DG$ weakly dominates under the Heavy Tail and Uniform instances but that $\TS$ is weakly dominant under Needle In Haystack is precisely due to the fact that the purposeful exploration engaged by $\TS$ in the early rounds leads it to suffer a relative reputational cost in the former case but not in the latter. Looking at the relative reputation plots in Figure \ref{relative_rep_plots}, we can interpret fixing a warm start $T_0$ as fixing the starting point on the relative reputation plots. The proportion of first rounds in the competition game that will go to a firm playing alg $A$ over a firm playing alg $B$ will correspond to the relative reputation proportion at time $T_0$. As a result, we see that for warm start $T_0 = 20$ on the relative reputation costly instances, $\DG$ will win the first agent in a larger proportion of the simulations than $\TS$. However, on the Needle In Haystack instances, we see that $\TS$ wins a larger proportion of the first agents compared to $\DG$.

What happens after the first agent? Looking at the EEOG values we note that, even though the game continues until $t=2000$, the game effectively ends much earlier than that. This seems to imply that when attempting to understand the performance of different algorithms in competition it is important to look at their performance for relatively small samples instead of asymptotic performance. This is due to the fact that the losing firm may enter into a ``death spiral" where the winner keeps learning and improving from its customers thus increasing its reputation while the losing firm gets starved of customers and progressively falls farther and farther behind its competitors. As a result, even though if $\TS$ could get sufficiently many agents it would eventually have a higher reputation than $\DG$, since it starts the game by losing agents to $\DG$ due to its lowered reputation from exploration in the warm start it ends up doing worse in the competition game.

% latex table generated in R 3.4.0 by xtable 1.8-2 package
% Thu Aug 16 13:13:01 2018
\begin{table}[ht]
\centering
\caption{Duopoly experiment for heavy tail instances}
\begin{tabular}{|c|c|c|c|}
  \hline
 & $T_0$ = 20 & $T_0$ = 250 & $T_0$ = 500 \\ 
  \hline
$\TS$ vs $\DG$ & \makecell{\textbf{0.29} $\pm$0.03\\ eeog \\ avg: 55\\ med: 0} & \makecell{\textbf{0.72} $\pm$0.02\\ eeog \\ avg: 570\\ med: 0} & \makecell{\textbf{0.76} $\pm$0.02\\ eeog \\ avg: 620\\ med: 98.5} \\ 
\hline  
  $\TS$ vs $\DEG$ & \makecell{\textbf{0.3} $\pm$0.03\\ eeog \\ avg: 37\\ med: 0} & \makecell{\textbf{0.88} $\pm$0.01\\ eeog \\ avg: 480\\ med: 0} & \makecell{\textbf{0.9} $\pm$0.01\\ eeog \\ avg: 570\\ med: 113.5} \\ 
\hline  
  $\DG$ vs $\DEG$ & \makecell{\textbf{0.62} $\pm$0.03\\ eeog \\ avg: 410\\ med: 7} & \makecell{\textbf{0.6} $\pm$0.02\\ eeog \\ avg: 790\\ med: 762} & \makecell{\textbf{0.57} $\pm$0.03\\ eeog \\ avg: 730\\ med: 608} \\ 
   \hline
\end{tabular}
\label{sim_ht}
\end{table}

\begin{table}[ht]
\centering
\caption{Duopoly experiment for needle in haystack instances}
\begin{tabular}{|c|c|c|c|}
  \hline
 & $T_0$ = 20 & $T_0$ = 250 & $T_0$ = 500 \\
  \hline
$\TS$ vs $\DG$ & \makecell{\textbf{0.64} $\pm$0.03\\ eeog \\ avg: 200\\ med: 27} & \makecell{\textbf{0.6} $\pm$0.03\\ eeog \\ avg: 370\\ med: 0} & \makecell{\textbf{0.64} $\pm$0.03\\ eeog \\ avg: 580\\ med: 121.5} \\
\hline  
  $\TS$ vs $\DEG$ & \makecell{\textbf{0.57} $\pm$0.03\\ eeog \\ avg: 150\\ med: 14} & \makecell{\textbf{0.52} $\pm$0.03\\ eeog \\ avg: 460\\ med: 78.5} & \makecell{\textbf{0.56} $\pm$0.02\\ eeog \\ avg: 740\\ med: 627.5} \\
  \hline
  $\DG$ vs $\DEG$ & \makecell{\textbf{0.46} $\pm$0.03\\ eeog \\ avg: 340\\ med: 128.5} & \makecell{\textbf{0.42} $\pm$0.02\\ eeog \\ avg: 650\\ med: 408} & \makecell{\textbf{0.42} $\pm$0.02\\ eeog \\ avg: 690\\ med: 466.5} \\
   \hline
\end{tabular}
\label{sim_nih}
\caption*{{The first line in each cell contains the average market share received by the firm playing Alg 1 (and the market share of Alg 2 is 1 - Alg 1 Market Share) as well as a 95\% confidence band. For example, the cell in the top left indicates that TS gets on average 64\% of the market when played against DG. The next line contain the average and median effective end of game for this set of simulations.}}
\end{table}
% latex table generated in R 3.4.0 by xtable 1.8-2 package
% Thu Aug 16 13:13:00 2018


\xhdr{Temporary Monopoly.} Recall that the incumbent firm enters the
market and serves as a monopolist until the entrant firm enters at
round $X$. \gaedit{For this section we focus on the warm start values and on the relative reputation costly instances where we observed that, under permanent duopoly, $\DG$ was the weakly dominant strategy. Concretely, our presented results are for $T_0 = 20$.} \footnote{However, we report the results for the Needle In Haystack instances in the supplementary
  material. The results are unsurprising, where $\TS$ is the dominant
  strategy for both the entrant and incumbent.}

\begin{finding}
  \textit{If the incumbent firm is a temporary monopolist for
    a sufficiently long time, it will deploy the better learning
    algorithm, $\TS$, even though, under the permanent duopoly, they
    would commit to $\DG$.}
\end{finding}

In Table \ref{ht_incum} we can see that, for $X = 200$, $\TS$ is a dominant strategy for the incumbent and that $\DG$ is the dominant strategy for the entrant on the heavy tail instances.\footnote{The result for the Uniform instances is the same, though it takes a higher $X$ to induce the incumbent to play $\TS$.} In the supplementary material we report the same experiment for different values of $X$ and show that our main result that, for sufficiently large $X$, $\TS$ is the dominant strategy for the incumbent is robust across the instances we consider.

% latex table generated in R 3.4.0 by xtable 1.8-2 package
% Wed Aug 15 19:03:20 2018

\begin{table}[ht]
\centering
\caption{Temporary Monopoly Heavy Tail X = 200}
\begin{tabular}{cc|c|c|c|}
  & \multicolumn{1}{c}{} & \multicolumn{3}{c}{Incumbent} \\
  & \multicolumn{1}{c}{} & \multicolumn{1}{c}{$\TS$}  & \multicolumn{1}{c}{$\DEG$}  & \multicolumn{1}{c}{$\DG$} \\\cline{3-5}
            & $\TS$ & \makecell{\textbf{0.003} \\$\pm$0.003} & \makecell{\textbf{0.083} \\ $\pm$0.02} & \makecell{\textbf{0.17} \\$\pm$0.02} \\ \cline{3-5}
Entrant  & $\DEG$ & \makecell{\textbf{0.045} \\$\pm$0.01} & \makecell{\textbf{0.25} \\$\pm$0.02} & \makecell{\textbf{0.23} \\$\pm$0.02} \\\cline{3-5}
            & $\DG$ & \makecell{\textbf{0.12} \\ $\pm$0.02} & \makecell{\textbf{0.36} \\$\pm$0.03} & \makecell{\textbf{0.3} \\$\pm$0.02} \\\cline{3-5}
\end{tabular}
\label{ht_incum}
\caption*{{The table contains the average market share for the entrant as well as 95\% confidence intervals. Note that smaller values in the table are better for the incumbent. Market shares are calculated as the fraction of users selecting a particular firm \textit{after} the entrant has already entered (i.e. the free rounds to the incumbent do not count towards the share)}}
\end{table}

Why do we observe that $\TS$ is the dominant strategy for the incumbent whereas in the permanent duopoly experiment we saw that, under the same parameters, $\DG$ was the weakly dominant strategy? The intuition for this is that competition in the duopoly forces the firms to worry about their reputation which dissuades them from committing to algorithms that involve pure exploration in the early rounds. One can view allowing one firm to temporarily be a monopolist as temporarily relaxing the ``incentive" component of exploration, exploitation, and incentives so that the incumbent firm only faces the classic tradeoff between exploration and exploitation. The incumbent only needs to worry about her reputation after $X$ periods when the entrant comes into the market and again needs to worry about incentivizing agents to select them over their competition. 

As a result, the incumbent is incentivized to commit to an algorithm that does exploration in the early rounds since she no longer suffers the same immediate relative reputational cost that she would suffer under competition as long as the $X$ is sufficiently large that she can begin to recover the reputational costs of exploration. Thus, counterintuitively, by having one firm be a monopoly and dominate the market, we can incentivize them to play $\TS$.

\end{document}
%%% Local Variables:
%%% mode: latex
%%% TeX-master: "../competing_bandits"
%%% End:
