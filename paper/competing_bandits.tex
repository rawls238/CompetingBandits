\documentclass[sigconf, anonymous, review,natbib=false]{acmart}

\usepackage{tikz} % SW: Added for plots
\usepackage{subfiles}
\usepackage{url}            % simple URL typesetting
\usepackage{booktabs}       % professional-quality tables
\usepackage{nicefrac}       % compact symbols for 1/2, etc.
\usepackage{microtype}      % microtypography
\usepackage{amsmath,amssymb,amsfonts}
%\usepackage{natbib} % cannot use with AAAI! A.S.

\usepackage{amsthm} % The amsthm package provides extended theorem environments
\usepackage{float}
\usepackage{sgame, tikz} % Game theory packages
\usepackage{caption}
\usepackage{algorithm,algpseudocode}
\usepackage{makecell}
 \usepackage{multirow}
 \usepackage{graphicx}
\theoremstyle{definition}
%\newtheorem{definition}{Definition}
\newtheorem{finding}{Finding}
%\newtheorem{conjecture}{Conjecture}
\newtheorem{puzzle}{Puzzle}
\captionsetup{font=footnotesize}

% Alex' macro
\usepackage{xspace}
\newcommand{\OMIT}[1]{}
\newcommand{\xhdr}[1]{\vspace{1mm} \noindent{\bf #1}}
\newcommand{\ie}{{\em i.e.,~\xspace}}
\newcommand{\eg}{{\em e.g.,~\xspace}}

\newcommand{\term}[1]{\ensuremath{\mathtt{#1}}\xspace}
\newcommand{\TS}{\term{TS}}
\newcommand{\DEG}{\term{DEG}}
\newcommand{\DG}{\term{DG}}
\newcommand{\Beta}{\term{Beta}} % for Beta distribution
\newcommand{\Eeog}{\term{EoG}} % shorthand for "effective end of game"

\newcommand{\eps}{\varepsilon}

\newcommand{\MRV}{mean reward vector\xspace} % mean reward vector
\newcommand{\MRVs}{mean reward vectors\xspace} % mean reward vector



% a very useful package for edits and comments, from David Kempe (USC)
\usepackage{color-edits}
%\usepackage[suppress]{color-edits}  % use this to suppress the package
\addauthor{as}{red}    % as for Alex
\addauthor{ga}{blue}  % ga for Guy
\addauthor{sw}{orange} % sw for Steven
% e.g. for Alex, provides \asedit{}, \ascomment{} and \asdelete{}.
\setlength{\tabcolsep}{8pt} % Default value: 6pt
\renewcommand{\arraystretch}{1.5} % Default value: 1

%drt24 hacks
% Letter paper
\setlength{\paperheight}{11in}
\setlength{\paperwidth}{8.5in}

% Hack to try to make acmart work with biblatex: https://tex.stackexchange.com/questions/37076/is-it-possible-to-load-biblatex-with-a-class-that-has-already-loaded-natbib
\let\citename\relax
%\RequirePackage[abbreviate=true, dateabbrev=true, isbn=true, doi=true, urldate=comp, url=true, maxbibnames=9, backref=false, backend=biber, style=ACM-Reference-Format, language=american]{biblatex}







\usepackage{booktabs} % For formal tables


% Copyright
%\setcopyright{none}
%\setcopyright{acmcopyright}
%\setcopyright{acmlicensed}
\setcopyright{rightsretained}
%\setcopyright{usgov}
%\setcopyright{usgovmixed}
%\setcopyright{cagov}
%\setcopyright{cagovmixed}


% DOI
\acmDOI{10.475/123_4}

% ISBN
\acmISBN{123-4567-24-567/08/06}

%Conference
\acmConference[WWW'19]{ACM WWW conference}{May 2019}{San Francisco, CA 2019}
\acmYear{2019}
\copyrightyear{2019}

\acmPrice{15.00}


\begin{document}
\title{Competing Bandits: \\ The Perils of Exploration under Competition}


\begin{abstract}
We empirically study the interplay between \textit{exploration} and \textit{competition}. Systems that learn from interactions with users
often engage in \emph{exploration}: making potentially suboptimal decisions in order to acquire new information for future decisions. However, exploration may hurt system's reputation in the near term, with adverse competitive effects. In particular, a system may enter a ``death spiral" when decreasing market share leaves the system with less users to learn from, which degrades system's performance relative to competition and further decreases the market share.

We ask whether better exploration algorithms are incentivized under competition. We run extensive numerical experiments in a stylized duopoly model in which two firms deploy multi-armed bandit algorithms and compete for myopic users.  We find that duopoly and monopoly tend to favor a primitive ``greedy algorithm" that does not explore, whereas a temporary monopoly (a duopoly with an early entrant) may incentivize better bandit algorithms. Our findings shed light on the ``first-mover advantage" in the digital economy.
\end{abstract}

%
% The code below should be generated by the tool at
% http://dl.acm.org/ccs.cfm
% Please copy and paste the code instead of the example below.
%
\begin{CCSXML}
<ccs2012>
<concept>
<concept_id>10010147.10010257.10010282.10010284</concept_id>
<concept_desc>Computing methodologies~Online learning settings</concept_desc>
<concept_significance>500</concept_significance>
</concept>
<concept>
<concept_id>10010147.10010257.10010258.10010261.10010272</concept_id>
<concept_desc>Computing methodologies~Sequential decision making</concept_desc>
<concept_significance>300</concept_significance>
</concept>
<concept>
<concept_id>10010405.10010455.10010460</concept_id>
<concept_desc>Applied computing~Economics</concept_desc>
<concept_significance>500</concept_significance>
</concept>
</ccs2012>
\end{CCSXML}

\ccsdesc[500]{Computing methodologies~Online learning settings}
\ccsdesc[300]{Computing methodologies~Sequential decision making}
\ccsdesc[500]{Applied computing~Economics}

\keywords{Bandits, Competition}


\maketitle

\subfile{content/introduction}

\subfile{content/model}

\subfile{content/perf_in_iso}

\subfile{content/inverted_u}

\subfile{content/barriers}

\subfile{content/revisited}

\subfile{content/conclusion}

\bibliographystyle{acm}
\bibliography{refs,bib-abbrv-short,bib-bandits,bib-AGT,bib-slivkins}


\end{document}