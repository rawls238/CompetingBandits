\documentclass[11pt,letterpaper]{article}

\usepackage{fullpage,times}

\usepackage[round]{natbib}

% a very useful package for edits and comments, from David Kempe (USC)
\usepackage{color-edits}
%\usepackage[suppress]{color-edits}  % use this to suppress the package
%\addauthor{ab}{green}   % ab for Ashwin
%\addauthor{bk}{magenta} % bk for Bobby
\addauthor{as}{red}      % as for Alex
%\addauthor{mb}{blue}    % mb for Moshe
% e.g. for Alex, provides \asedit{}, \ascomment{} and \asdelete{}.

\begin{document}
%\vspace{-30mm}

\section*{Submission: Competing Bandits: The perils of Competition\\
{\large (by Guy Aridor, Aleksandrs Slivkins and Zhiwei Steven Wu)}\\
{\large Workshop: Learning in the Presence of Strategic Behavior}
}

We empirically study the interplay between \textit{exploration} and
  \textit{competition}. Systems that learn from interactions with
  users often engage in \emph{exploration}: making potentially
  suboptimal decisions in order to acquire new information for future
  decisions. However, when multiple systems are competing for
    the same market of users, exploration may hurt a system's
    reputation in the near term, with adverse competitive effects. In particular, a system may enter a ``death spiral", when the short-term reputation cost decreases
    the number of users for the system to learn from, which degrades the
    system's performance relative to competition and further decreases
    the market share.

We ask whether better exploration algorithms are incentivized under competition. We run extensive numerical experiments in a stylized duopoly model in which two firms deploy multi-armed bandit algorithms and compete for myopic users.  We find that duopoly and monopoly tend to favor a primitive ``greedy algorithm" that does not explore and leads to low consumer welfare, whereas a temporary monopoly (a duopoly with an early entrant) may incentivize better bandit algorithms and lead to higher consumer welfare. Our findings shed light on the first-mover advantage in the digital economy by exploring the role that data can play as a barrier to entry in online markets.

\vspace{2mm}

This paper continues the literature on ``exploration and incentives", which in turn belongs to the area of ``learning in the presence of strategic behavior". It is an experimental counterpart to \citet{CompetingBandits-itcs18}, which has initiated the study of ``exploration and competition" and obtained obtained a number of theoretical results with ``asymptotic" flavor. While they considered a similar duopoly model and arrived at similar high-level conclusions, their details and assumptions are substantially different from ours for the sake analytical tractability, and their theorems have no direct bearing on our simulations. In comparison, we provide a more nuanced and ``non-asymptotic" perspective, looking for substantial effects within relevant time scales. We plan a merged journal submission to a venue in economics or operations research.

\begin{small}
\bibliographystyle{plainnat}
\bibliography{bib-abbrv,bib-slivkins}
\end{small}


\end{document}
